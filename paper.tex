% SIAM Article Template
\documentclass[review]{siamart1116}

% Information that is shared between the article and the supplement
% (title and author information, macros, packages, etc.) goes into
% ex_shared.tex. If there is no supplement, this file can be included
% directly.

% SIAM Shared Information Template
% This is information that is shared between the main document and any
% supplement. If no supplement is required, then this information can
% be included directly in the main document.


% Packages and macros go here
\usepackage{lipsum}
\usepackage{amsfonts}
\usepackage{graphicx}
\usepackage{epstopdf}
\usepackage{algorithmic}
\usepackage[nolist,nohyperlinks]{acronym}
\usepackage{graphicx}
\usepackage{tikz}
\usepackage{grffile}
\usepackage{xr}
\usepackage{subcaption}


\ifpdf
  \DeclareGraphicsExtensions{.eps,.pdf,.png,.jpg}
\else
  \DeclareGraphicsExtensions{.eps}
\fi

%strongly recommended
\numberwithin{theorem}{section}

% Declare title and authors, without \thanks
\newcommand{\TheTitle}{Intrinsic Isometric Manifold Learning with Application to Localization} 
\newcommand{\TheAuthors}{A. Schwartz and R. Talmon}

% Sets running headers as well as PDF title and authors
\headers{Intrinsic Isometric Manifold Learning}{\TheAuthors}

% Title. If the supplement option is on, then "Supplementary Material"
% is automatically inserted before the title.
\title{{\TheTitle}\thanks{Submitted to the editors DATE.
\funding{This work was funded by the Fog Research Institute under contract no.~FRI-454.}}}

% Authors: full names plus addresses.
\author{
  Ariel Schwartz\thanks{Viterbi Faculty of Electrical Engineering, Technion – Israel Institute of Technology, Israel
    (\email{ariels@technion.campus.ac.il}}
  \and
  Ronen Talmon\thanks{Viterbi Faculty of Electrical Engineering, Technion – Israel Institute of Technology, Israel  (\email{ronen@ee.technion.ac.il}}
}

\usepackage{amsopn}
\DeclareMathOperator{\diag}{diag}


\newcommand\plotresults	[1]{
	\begin{figure}[h]	
	\begin{centering}
		\begin{subfigure}[b]{0.4\linewidth}
			\includegraphics[width=1\linewidth]{#1+{/intrinsic}}
			\captionsetup{justification=centering}
			\caption{Intrinsic space}
		\end{subfigure}
		\hfill
		\begin{subfigure}[b]{0.50\linewidth}
			\includegraphics[width=1\linewidth]{#1+/observed}
			\caption{Observed space}
		\end{subfigure}
	\end{centering}
	\begin{centering}
		\begin{subfigure}[b]{0.45\linewidth}
			\includegraphics[width=1\linewidth]{#1+/metric_local_dense}
			\caption{Intrinsic metric}
		\end{subfigure}
		\hfill
		\begin{subfigure}[b]{0.45\linewidth}
			\includegraphics[width=1\linewidth]{#1/dist_local}
			\caption{Intrinsic distance approximation}
		\end{subfigure}
	\end{centering}
	\begin{centering}
		\begin{subfigure}[b]{0.32\linewidth}
			\includegraphics[width=1\linewidth]{#1/standard_isomap_embedding}
			\captionsetup{justification=centering}
			\caption{Standard Isomap  \\Embedding}
		\end{subfigure}
		\hfill
		\begin{subfigure}[b]{0.32\linewidth}
			\includegraphics[width=1\linewidth]{#1/embedding_intrinsic_isomap_dense}
			\captionsetup{justification=centering}
			\caption{Intrinsic Isomap \\Embedding}	
		\end{subfigure}
		\hfill
		\begin{subfigure}[b]{0.32\linewidth}
			\includegraphics[width=1\linewidth]{#1/embedding_intrinsic_isometric_dense}
			\captionsetup{justification=centering}
			\caption{Intrinsic Isometric \\Embedding}
		\end{subfigure}
	\end{centering}
	\begin{centering}
		\begin{subfigure}[b]{0.32\linewidth}
			\includegraphics[width=1\linewidth]{#1/standard_isomap_stress}
			\captionsetup{justification=centering}
			\caption{Standard Isomap \\Stress}
		\end{subfigure}
		\hfill
		\begin{subfigure}[b]{0.32\linewidth}
			\includegraphics[width=1\linewidth]{#1/stress_intrinsic_isomap_dense}
			\captionsetup{justification=centering}
			\caption{Intrinsic Isomap \\Stress}
		\end{subfigure}
		\hfill
		\begin{subfigure}[b]{0.32\linewidth}
			\includegraphics[width=1\linewidth]{#1/stress_intrinsic_isometric_dense}
			\captionsetup{justification=centering}
			\caption{Intrinsic Isometric \\Stress}
		\end{subfigure}
	\end{centering}
	\caption{Punctured fishbowl}
\end{figure}
}




%%% Local Variables: 
%%% mode:latex
%%% TeX-master: "ex_article"
%%% End: 


% Optional PDF information
\ifpdf
\hypersetup{
	pdftitle={\TheTitle},
	pdfauthor={\TheAuthors}
}
\fi

% The next statement enables references to information in the
% supplement. See the xr-hyperref package for details.

\externaldocument{ex_supplement}

% FundRef data to be entered by SIAM
%<funding-group>
%<award-group>
%<funding-source>
%<named-content content-type="funder-name"> 
%</named-content> 
%<named-content content-type="funder-identifier"> 
%</named-content>
%</funding-source>
%<award-id> </award-id>
%</award-group>
%</funding-group>

\begin{document}
	\begin{acronym}
		\acro{ADAM}{ADAptive Moment estimation}
		\acro{ANN}{Artificial Neural Network}
		\acro{AWGN}{Additive White Gaussian Noise}
		\acro{EDM}{Euclidean Distance Matrix}
		\acro{EDMCP}{Euclidean Distance Matrix Completion Problem}
		\acro{GMM}{Gaussian Mixture Model}
		\acro{LLE}{Locally Linear Embedding}
		\acro{LS-MDS}{Least Squares Multi Dimensional Scaling}
		\acro{MDS}{Multi Dimensional Scaling}
		\acro{MVU}{Maximum Variance Unfolding}
		\acro{PCA}{Principal Component Analysis}
		\acro{RSS}{Received Signal Strength}
		\acro{SDE}{Stochastic Differential Equation}
		\acro{SDP}{SemiDefinite Programming}
		\acro{SMACOF}{Scaling by MAjorizing a COmplicated Function}
		\acro{SVD}{Singular Value Decomposition}
	\end{acronym} 
	
	\maketitle
	
	% REQUIRED
	\begin{abstract}
		Data lying on manifolds commonly appear in machine learning, signal processing and data analysis. It is often the case that although these manifolds are inherently low-dimensional, they are embedded in an ambient space of a much higher dimension. Manifold learning techniques aim to recover the low-dimensional structure of the data hidden in the observed data samples. This is often accomplished by building alternative low-dimensional representations of the data while preserving, as much as possible, some of its observed properties which are usually related to the geometric structure of the observed data samples. In this paper, we propose to deviate from this standard practice. Instead of learning the structure of the observed manifold, we view the observed data only as a proxy to a latent unobserved intrinsic manifold for which build a new geometry preserving representation. In particular, we propose a new manifold learning method that uses the push-forward metric, between the latent intrinsic manifold and the observed manifold, in order to compute geometric properties on the observed data manifold as if they were computed directly on the underlying latent intrinsic manifold. We show that using this intrinsic metric, it is possible to learn an intrinsic and isometric data representation which respects the latent manifold geometry. We further propose a method for robust estimation of the push-forward metric from observed data without knowledge of the system model using artificial neural networks. Finally, we show that the proposed approach can be successfully applied to the problem of indoor localization in ad-hoc sensor networks of unknown type. In particular, we demonstrate that we can discover the physical space and infer a non-linear function which maps sensor measurements to a location in the physical space allowing us to localize an agent, all without knowing a-priori the nature of the connection between the agent location and the sensor measurements.
	\end{abstract}
	
	% REQUIRED
	\begin{keywords}
		manifold learning, geodesic distance, inverse problem, metric estimation, intrinsic, isometric, positioning, sensor invariant
	\end{keywords}
	
	% REQUIRED
	\begin{AMS}
		??????
	\end{AMS}
	

	\externaldocument{paper.tex}

\section{Introduction}
	\label{sec:introduction}

	Making measurements has always been an integral part of the scientific method. It is through measurements and observations of various physical quantities that we can explore the surrounding world, allowing us to understand, make conclusions and even predictions about systems of interest. However, although measurements have such an integral role in our lives, we are usually not directly interested in the measurements themselves but rather in the properties or the state of a latent system which generates or causes these observed measurements, a system which might not be directly observable. For example when considering a radar system, we are not truly interested in the pattern of the electromagnetic wave received and measured by the antenna but rather in the location, size and velocity of an object in the antennas range whose reflectance shaped this wave.

	A latent system of interest can sometimes be observed in several different ways and data collected from it can be represented in various forms. In such cases we find that the specific observation and data representation methods used have a large influence on how effectively and easily we can make inferences about the latent system from the observed data. Ideally, the observation data should be represented in way in which its relation to the latent system is well understood and is as simple as possible, allowing it to be easily related to the actual system of interest. Not only the complexity of the connection between a system and its observations plays a role in the ability to analyze data but also the dimensionality of the observations. It is often the case that systems which are governed by a small number of variables (and are hence inherently low-dimensional in nature) are observed via redundant high-dimensional representations, producing data that lies on a low-dimensional manifold embedded in a high-dimensional ambient measurement space. Such unnecessarily high-dimensional representations obscure the low-dimensional nature of the system of interest and needlessly increase the computational cost of any algorithms applied to the observed data.
		
	When trying to tackle any signal processing or machine learning task, a considerable amount of focus is put initially on choosing or building an appropriate “convenient” representation of the observed data which would ideally be low-dimensional and simply relatable to the system of interest. This results in domain-specific knowledge being used to design hand-tailored and task-specific data representations, each used for a specific problem domain. While this practice has worked in some fields, it time consuming, tedious and sometimes impossible to follow due to insufficient pre-existing understanding of the latent system and its connection to the observed data. The availability of data generated by systems which are not sufficiently understood has become a commonly occurring situation with the massive increase in the ability to acquire, store and share data, leading to the rise of the concept of “Big-Data”. This trend has led to a reversal in the paradigm of tailoring data representations to a specific task based on prior knowledge, and nowadays the starting point for many interesting tasks is the existence of a large amount of data or measurements generated by systems which are not well understood. Much research in the fields of artificial intelligence, machine learning, data science and data mining has be devoted to methods for automatic learning directly from observed data in order to gain insights about the latent system. Representation learning and manifold learning methods \cite{bengio2013representation} attempt to learn representations from the available data, trying to lower the dimensionality of the data and uncover properties of the latent generating system. These methods have been proven to have the potential of improving existing data representations and discovering new representations in domains where little or no pre-existing knowledge is available. 

	Papers and work regarding representation and manifold learning learning usually address the relationship between the observed data and the structure of the representation learned from them. It can generally be said that most methods try to calculate some properties of the available observed data structure (these are often geometric properties) and then attempt to preserve them in the new learned representation \cite{tenenbaum2000global, perraultriemannian, bengio2013representation, coifman2006diffusion, donoho2003hessian, kamada1989algorithm, roweis2000nonlinear, saul2003think, singer2008non, tipping1999probabilistic}. This approach naturally arises from the desire to “lose” as little as possible information form the original data-set. This is especially clear for auto-encodes which explicitly require the the original observed data be completely retrievable from the new generated representation \cite{vincent2008extracting, vincent2010stacked, hinton2006reducing}. As a result of this approach which values the observed data over everything else, a considerable amount of effort is being invested in preservation of properties of the observed data. Unfortunately the way in which data is initially presented is often quite arbitrary and without a clear known connection to the latent system of actual interest. This becomes immediately clear if one considers the possibility that same system can produce very different observed data structures when using different measurement modalities. In light of this, it is sometimes clearly undesirable to invest such effort in preserving observed properties which might be arbitrary and irrelevant and one should instead try to seek and preserve intrinsic properties of the data, properties which are inherent to the latent system of interest and will manifest in any reasonable measurement or representation of the system. Unfortunately, most existing manifold learning methods do not address the issue of the relationship between the structure of the learned representation and some latent intrinsic structure or meaningful phenomena exhibited system of interest which generated the observed measurements. 
		
	This paper is organized as follows: In \cref{sec:motivation} we present a toy problem which serves as motivation for intrinsic and isometric manifold learning, in \cref{sec:setting} we formulate the problem mathematically. In \cref{sec:Intrinsic-isometric-manifold-learning} we present a novel manifold learning method, which is both intrinsic and isometric thus uncovering the geometric structure of the latent system of interest, however, we will notice that this methods requires knowledge of some local properties of the unknown observation function. In \cref{sec:Intrinsic-Metric-Estimation} we present a method to exploit observed measurements in order to robustly estimate the local properties of the observation function that were needed for the algorithm presented in \cref{sec:Intrinsic-isometric-manifold-learning}. In \cref{sec:results} we present the results of our suggested algorithm on a few synthetic data sets and also revisit the motivating example described in \cref{sec:motivation} and show that our approach indeed allows for intrinsic-isometric manifold learning in realistic conditions, enabling model invariant positioning and mapping. In \cref{sec:conclusions} we conclude the work, discuss a few key issues and a few possible future research directions.

\section{Toy problem and motivation}
	\label{sec:motivation}
	
	Our motivating example is that of mapping a 2-dimensional region and positioning of an agent within it base on observations made by the agent. Consider a 2-dimensional region which might represent the interior of a structure. We denote the set of points belonging to this region by $\mathcal{X}$ and the location of the agent by $\mathbf{x}$ as illustrated in \cref{fig:An-agent-in}.
		
	\begin{figure}[h]
		\begin{centering}
			\includegraphics[width=0.7\textwidth]{figures/Chapter_1/loc_example_intrinsic_shape}
			\par\end{centering}
		\caption{Agent in a $2$-dimensional space\label{fig:An-agent-in}}
	\end{figure}
	
	We are not able to directly observe the shape of $\mathcal{X}$ or the location of the agent $\mathbf{x}\in\mathcal{X}$, therefore they represent a latent system and variable. At each point $\mathbf{x}\in\mathcal{X}$ we can make measurements or observations which are functions of the location $\mathbf{x}$ at which the measurement was taken (for now we exclude the possibility of noise or other influencing variables). To allow visualization of this problem, we assume that 3 different measurements are made at each 2-dimensional location as described in. One might for example measure \ac{RSS} values at $\mathbf{x}$ from different transmitting antennas,a model that has been considered many times for the propose of internal positioning and localization systems \cite{ash2004sensor, bal2009localization, boukerche2007localization, costa2006distributed, kotaru2015spotfi, moses2003self, niculescu2003ad, patwari2003relative, ramadurai2003localization, yang2009indoor}.For the sake of simplicity and intuition, we further restrict ourselves to measurements decay as the agent is further away from a transmitting station (as is the case in the “free-space” model which corresponds to outdoor signal propagation). Such measurements are visualized in \cref{fig:First-measurement-modality} where the antenna symbol represent the location from which the signal originates and different colors represent different \ac{RSS} measurement values. One can also consider another set of 3 such measurements for which no known model is available as visualized \cref{fig:Second-measurement-modality}.


	\begin{figure}
		\centering
		\begin{subfigure}{.5\textwidth}
			\centering
			\includegraphics[width=.5\linewidth]{figures/Chapter_1/sensor_1_all}
			\caption{First measurement modality\label{fig:First-measurement-modality}}
		\end{subfigure}%
		\begin{subfigure}{.5\textwidth}
			\centering
			\includegraphics[width=.5\linewidth]{figures/Chapter_1/sensor_2_all}
			\caption{Second measurement modality\label{fig:Second-measurement-modality}}
		\end{subfigure}
			\caption{Two different observation function modality\label{fig:test}}
	\end{figure}

		
		
	Although the system in both cases is observed via 3 different measurement values it is intuitively clear that it only has 2 inherit degrees of freedom, corresponding to the 2-dimensional location coordinates of the agent. If we look at many measurements taken at different locations, we get the structure of a 2-dimensional manifold embedded in 3-dimensional observation space as visualized in \cref{fig:Creation-of-observed} for the case of the \ac{RSS} measurements.
		
	\begin{figure}[h]
		\begin{centering}
			\includegraphics[width=0.7\textwidth]{figures/Chapter_1/measurment_setting}
			\par\end{centering}
		\caption{The observed manifold\label{fig:Creation-of-observed}}
	\end{figure}
		

	If sensor measurements are unique in $\mathcal{X}$ (i.e. no two locations within $\mathcal{X}$ correspond to the same set measurement values) one can ask whether we can infer the location $\mathbf{x}$ at which a certain measurement was made given its respective measurement value $\mathbf{y}=f\left(\mathbf{x}\right)$. Such inference would provide a localization of the agent based on the observed measurements. Applying this localization to many measurements obtained from different locations in $\mathcal{X}$ would provide a mapping of $\mathcal{X}$. If $f$ is known and the connection between the location and the measurement values is understood this becomes a classical problem of model-based localization, which is a specific case of a broader class of non-linear inverse problems \cite{engl2005nonlinear}. In such problems, the understanding of system is exploited in order to inverse it and retrieve the latent states of the system.
		
	However, as is often the case with signal propagation in indoor settings, the exact connection between the measurement values and the location of the agent is either unknown or is too complicated due to dependence on many factors which are unknown to us beforehand (such as room geometry, locations of walls, reflection, transmission, absorption of materials, etc.). As a result we cannot usually assume that the observation model $\mathbf{y}=f\left(\mathbf{x}\right)$ is known. This leads to a more challenging question: would it be possible to retrieve $\mathbf{x}$ from $f\mathbf{\left(\mathbf{x}\right)}$ without knowing $f$; we will call this problem “blind localization” or a “blind inverse problem”.
		
	Given that we find ourselves interested in the 2-dimensional structure of a set of measurements embedded in a 3-dimensional ambient space, one could typically try to address this as a manifold learning or dimensionality reduction problem. Unfortunately, application of manifold learning algorithms gives somewhat disappointing results as can be seen in \cref{fig:Application-of-manifold}. While some of these methods provide an interpretable low-dimensional parameterization of the latent space by preserving the basic topology of the manifold (points close to each other in the physical 2-dimensional space remain close to each other in the learned representation), non of them recover the true location of the agent or a proper structure preserving mapping of the intrinsic space.
	
	\begin{figure}[h]
		\centering{}%
		\begin{minipage}[t]{1\columnwidth}%
			\begin{center}
				\includegraphics[width=1\textwidth]{figures/Chapter_1/ml_total}
				\par\end{center}%
		\end{minipage}\caption{Manifold learning for ``blind'' localization and mapping \label{fig:Application-of-manifold}. The first row corresponds to the application on the manifold created via the observation function visualized in \cref{fig:First-measurement-modality}. The second row corresponds to the application on the manifold created via the observation function visualized in \cref{fig:Second-measurement-modality}. The bottom row represents application of manifold learning algorithm directly to the intrinsic manifold}
	\end{figure}

	This is even more evident if the same system is observed again via a set of measurements of a different nature. The measurement functions used for this second observation modality are visualized in. Here, the measurements do not have an intuitive interpretation of a possible physical measurement which better simulates situation where the measurement model is unknown.
		
		
	We notice that we get very different results for each method when compared to the results in \cref{fig:Application-of-manifold}. It is therefore clear that these methods learn and preserve the structure of the observed-measurement manifold but they fail to capture the structure of the intrinsic latent system “behind” the measurements. Not only does this fail to achieve our goal of localization, it also emphasizes the dependence of manifold learning on the way data is observed/measured and that different observations of the same latent system lead to different learned low dimensional representation. Even if we apply manifold learning methods directly to the latent variable $\mathbf{x}\in\mathcal{X}$ we still do not necessarily retrieve the structure of $\mathcal{X}$ as seen in \cref{fig:Application-of-manifold}. This is because not all manifold learning methods are isometric ,i.e., they do not preserve distances.	
		
	This example shows the inadequacy of existing manifold learning algorithms for solving problems where we are interested is recovering the intrinsic geometric structure of a latent system. 
	
	Existing manifold learning algorithms generate new representations of data manifolds while preserving cretin properties of the \textit{observed} manifold. This approach initially seems quite reasonable; however, as was shown in the motivating example given in \cref{sec:motivation}, this might not always be the best strategy. When data originates from an intrinsic low-dimensional latent system observed via an observation function, the observed manifold is affected by the specific (and often arbitrary) observation function used, as a result, the use of different observation modalities results in different generated representations when applying manifold learning methods. This fails to capture the intrinsic characteristics of the different observed manifolds which all originate from the same latent low-dimensional system. Such settings give rise to the need for manifold learning algorithms which are \textit{intrinsic}. Intrinsic in this context means that the learned representation should not depend on the observation function or sensor modality used. However, intrinsicness by itself is not enough in order to retrieve the latent low-dimensional geometric structure of the data and ,as the motivating example also showed, even if applied directly to the low-dimensional latent space (thus avoiding any dependence on a specific observation function), many existing manifold learning methods distort the geometric structure of the latent low-dimensional space, which in many cases bears a special significance and might be relatable to some meaningful physical properties. In order to explicitly preserve the geometric structure of the latent manifold, the learned representation needs to be \textit{isometric} (distance preserving) as well. 
		
	A manifold learning method, which is isometric with respect to the intrinsic structure hidden in data, would allow for the retrieval of $\mathbf{x}$ and $\mathcal{X}$ from observed data without requiring knowledge of the specific observation model. In the case of the localization and positioning this would be in contrast to existing internal positioning algorithms which mostly depend on developing complex models for indoor signal propagation.
	
	In this paper our goal is to present an approach for dimensionality reduction which achieves these two goals simultaneously, we do this by developing a manifold learning method that is \textit{isometric with respect to the latent intrinsic geometric structure}.
	
	
\section{Problem formulation}
	\label{sec:setting}

	Let ${\cal X}$ be a path-connected (i.e. each two points in $\mathcal{X}$ can be connected via a curve that does not leave $\mathcal{X}$) subset of $n$-dimensional Euclidean space $\mathbb{R}^{n}$. $\mathcal{X}$ is the intrinsic or latent manifold, $\mathbb{R}^{n}$ the intrinsic vector space and n the intrinsic dimension. Let $f:{\cal X}\rightarrow\mathbb{R}^{m}$ be a continuously-differentiable, injective function called the observation function. The observed manifold, which is the image of f when applied to ${\cal X}$, is denoted by ${\cal Y}=f\left({\cal X}\right)$. $\mathbb{R}^{m}$ is called the observation vector space and m the observation dimension. Since f is injective we have that $n \leq m$ (for many cases of interest involving high-dimensional observations we might even have that $n\ll m$).
		
	The intrinsic manifold ${\cal X}$ represents the set of all the possible latent states of a low-dimensional system. These states are not directly accessible (which justifies the use of the term latent) but are rather only indirectly observable via f which represents the measurements we are given for each intrinsic state ${\bf x}\in{\cal X}$. ${\cal Y}$ represents the set of all possible measurement/observation values on $\mathcal{X}$. In any realistic scenario we can only discuss finite sets of sample points from both $\mathcal{X}$ and ${\cal Y}$, therefore we also define sample subsets of these two manifolds: Let ${\cal X}_{s}=\left\{ {\bf x}_{i}\right\} _{i=1}^{N}$ be the intrinsic subset of N points sampled from $\mathcal{X}$, such that for any integer $i$ satisfying $1\leq i\leq N$ we have ${\bf x}_{i}\in{\cal X}_{s}\subseteq{\cal X}$. Points in ${\cal X}_{s}$ are denoted by ${\bf x}_{i}=\left(\begin{array}{ccccc} x_{i,1} & x_{i,2} & \ldots & \ldots & x_{i,n}\end{array}\right)$ where $x_{i,j}$ is the $j$-th entry of the $i$-th member of the the set ${\cal X}_{s}$. Analogously, ${\cal Y}_{s}=f\left({\cal X}_{s}\right)$ is the subset of all observation function values received by observing the intrinsic sampled subset ${\cal X}_{s}$. We call ${\cal Y}_{s}$ the observation sampled subset. Points in ${\cal Y}_{s}$ are denoted by $f\left(\mathbf{x}_{i}\right)={\bf y}_{i}=\left(\begin{array}{ccccc} y_{i,1} & y_{i,2} & \ldots & \ldots & y_{i,m}\end{array}\right)$ and are called the observed variable, where $y_{i,j}$ is the j-th entry of the i-th member of the the set ${\cal Y}_{s}$. We do not assume that the observation function $f$ is known
		
	To make the above terms more intuitive we present them with respect to the localization example given in \cref{sec:introduction} which will be discussed again in \cref{ssec:localization} . In this example the intrinsic vector space is the $2$-dimensional physical plane which is of intrinsic dimension $n=2$ and $\mathcal{X}$ represents the set of all locations that our agent can travel to within $\mathbb{R}^{2}$. f represents the observations we are able to make at each location, which depend solely on the location where the measurement was made. $\mathcal{Y}$ represents the set of all possible observation values. The requirement that $f$ be injective is clear in this context if we want to be able to map each measurement value to a unique corresponding location of the agent. 
		
	Under the setting just describe and given access only to the observation sampled subset $\mathcal{Y_{S}}$, we wish to generate a new embedding ${\cal \widetilde{X}}_{s}=\left\{ \tilde{{\bf x}}_{i}\right\} _{i=1}^{N}$ of the observed points ${\cal Y}_{s}=\left\{ {\bf y}_{i}\right\} _{i=1}^{N}$ into $n$-dimensional Euclidean space, which respects the Euclidean metric structure of the latent set $\mathcal{X}_{s}$ which can be viewed as a metric space with respect to the Euclidean distance in the intrinsic vector space $\mathbb{R}^{n}$:
	\[
	d_{euc}\left(\text{\ensuremath{\mathbf{x}}}_{i},\text{\ensuremath{\mathbf{x}}}_{j}\right)=\left\Vert {\bf x}_{i}-{\bf x}_{j}\right\Vert 
	\]	
	In order to give a quantitative measure of ``structure preservation'' and ``intrinsic-isometry'' of our constructed embedding, we utilize the well known stress function which is commonly used in \ac{LS-MDS}. This function penalizes for discrepancies between the inter-point Euclidean distances in the constructed embedding and some ideal distance or dissimilarity. In our case the ideal distances are the true Euclidean distances in the intrinsic space. This results in the following cost function for the embedding process:
	\begin{equation}
	\sigma_{stress}\left({\cal \widetilde{X}}_{s}\right)=\sum_{i<j}\left(\left\Vert \mathbf{\widetilde{x}}_{i}-\mathbf{\widetilde{x}}_{j}\right\Vert -\left\Vert {\bf x}_{i}-{\bf x}_{j}\right\Vert \right)^{2}\label{eq:stress}
	\end{equation}
	Lower stress values imply that the Euclidean structure of the embedding respects the intrinsic Euclidean geometry more. Our goal is to find a set of embedding points ${\cal \widetilde{X}}_{s}=\left\{ \tilde{{\bf x}}_{i}\right\} _{i=1}^{N}$that minimizes this cost function. We notice that the stress cost function depends only on inter-point distances and is therefore invariant to rigid rotations of the created embedding. This means that even in the best case scenario we can only expect to retrieve ${\cal X}_{s}$ up to an unknown rigid rotation. This however is insignificant in most practical cases since it is equivalent to an arbitrary rotation of the problem axis which does not have a practical effect on any signal processing or machine learning task.
	Although the values of observed measurements $\mathcal{Y_{S}}$ are known, the connection between them and the intrinsic states is hidden since the observation function $f$ is unknown, hence the structure of $\mathcal{X}_{S}$ cannot be directly inferred from $\mathcal{Y}_{S}$ therefore the main challenge in the goal we have presented is that the ground-truth inter-point Euclidean distances $\left\Vert {\bf x}_{i}-{\bf x}_{j}\right\Vert $, which we are trying to adhere to, are unknown and need to be approximated from the observed data. A second challenge is that the cost function presented in \cref{eq:stress} leads to a non-convex optimization problem which requires a ``good'' initial solution in order to converge to a local-minima which is close to the global minima.
	\externaldocument{paper.tex}

\section{Proposed algorithm - Intrinsic Isometric Manifold Learning}
	\label{sec:Intrinsic-isometric-manifold-learning}

	
	\subsection{Algorithm high-level overview}
	\label{ssec:Algorithm-high-level-overview}
	
	First we will describe an Intrinsic Isometric manifold learning algorithm which operates assuming that the values of the matrix $\mathbf{M}\left({\bf y}_{i}\right)=\frac{df}{dx}\left(f^{-1}\left({\bf y}_{i}\right)\right)\frac{df}{dx}\left(f^{-1}\left({\bf y}_{i}\right)\right)^{T}$ are given for all $y_{i}\in{\cal Y}_{s}$, where $\frac{df}{dx}\left(\mathbf{x}\right)$ is the Jacobian of the observation function w.r.t the intrinsic variable at the intrinsic point ${\bf x}_{i}$. Although this is a very restrictive requirement, we show in the \cref{sec:Intrinsic-Metric-Estimation} how it can, in some scenarios, be robustly approximated from the accessible sample subset ${\cal Y}_{s}$ without explicit knowledge of $f$.
	
	Our approach then for achieving the goal described is \cref{sec:setting} is implemented by first recognizing that $\mathbf{M}\left({\bf y}_{i}\right)^{\dagger}=\left[\frac{df}{dx}\left(f^{-1}\left({\bf y}_{i}\right)\right)\frac{df}{dx}\left(f^{-1}\left({\bf y}_{i}\right)\right)^{T}\right]^{\dagger}$ is in fact the push-forward metric tensor and the observation function $f$ creates a isomorphism between the intrinsic and observed manifolds when using this push-forward metric as the Riemannian metric on the tangent planes of the observed manifold $\mathcal{Y}$. This isomorphism allows us to calculate approximation of short range intrinsic Euclidean distances on the observed manifold in a way that is invariant to the observation function $f$, as if they were calculated directly on the latent intrinsic manifold $\text{\ensuremath{\mathcal{X}}}$. We then construct an embedding of the observed manifold into $n$-dimensional Euclidean space by performing \ac{LS-MDS} using the approximations of the short-range intrinsic distances as the target distances. Since only short-range distances are approximated, we use a weighted version of the stress function (presented in Equation \cref{eq:w-intrinisc_stress}) which can ``ignore'' un-approximated inter-point distances and call it the \textit{partial stress function}. As this defines a non-convex cost function we are required to provide the optimization process with a good initial solution, to do so we recognize that using intrinsic short-range distances we can calculate approximations of all intrinsic geodesic distances on the manifold. With this we define an intrinsic version of the well known Isomap algorithm \cite{tenenbaum2000global}, which attempts to embed the sampled points in an Euclidean space so that they respect the approximated geodesic distances. The resulting embedding is used as an initial embedding for the minimization of the partial stress function \cref{eq:w-intrinisc_stress}. Due to the low-dimension constraint imposed on the embedding, the minimization of the partial stress function leads also to minimization of the full unweighted stress function in \cref{eq:stress}. The flow of this algorithm is illustrated in \cref{fig:Intrinsic-isometric-manifold-lea}.
	
	This approach harnesses the ability of eigen-decomposition to give a globally optimal solution to the embedding problem given all inter-point distances and the ability of the \ac{LS-MDS} optimization approach to only consider local intrinsic Euclidean structure. Convergence of the optimization process is achieved provided that the embedding received via eigen-decomposition is ``close enough'' to the true intrinsic structure. Since, as we will show, this might not be the case for highly non-convex intrinsic manifolds we also devise an iterative multi-scale optimization scheme to avoid such occurrences. 
	
	\begin{figure}[h]
		\begin{centering}
			\includegraphics[width=1\textwidth]{figures/Chapter_3/flow}
			\par\end{centering}
		\caption{Intrinsic-isometric manifold learning flow graph\label{fig:Intrinsic-isometric-manifold-lea}}
	\end{figure}
	
	\subsection{Local intrinsic geometry approximation}
	\label{ssec:Intrinsic-geometry-approximation}
	
	Minimization of the cost function given in \cref{sec:setting} first requires an approximation of the intrinsic Euclidean inter-point distances $d_{i,j}=d_{euc}\left(\text{\ensuremath{\mathbf{x}}}_{i},\text{\ensuremath{\mathbf{x}}}_{j}\right)=\left\Vert {\bf x}_{i}-{\bf x}_{j}\right\Vert $ without having direct access to the latent intrinsic variables themselves. To overcome this obstacle we recognize that ${\cal X}$ can be viewed as a Riemannian manifold by endowing it with the dot-product inherited from its $n$-dimensional intrinsic Euclidean space, and that $\mathbf{M}\left({\bf y}_{i}\right)^{\dagger}=\left[\frac{df}{dx}\left(f^{-1}\left({\bf y}_{i}\right)\right)\frac{df}{dx}\left(f^{-1}\left({\bf y}_{i}\right)\right)^{T}\right]^{\dagger}$ gives us the push-forward metric tensor on $\mathcal{Y}$ with respect to the transformation between $\mathcal{X}$ and $\mathcal{Y}$ (the observation function). With this, an isomorphism is established between the two manifolds $\mathcal{X}$ and $\mathcal{Y}$ which enables
	us to use $\mathcal{Y}$ as a proxy for making intrinsic geometric calculations on the latent manifold $\mathcal{X}$. A similar observation was made in \cite{singer2008non}, where the following approximation was suggested for calculating intrinsic inter-point distances:
	\begin{equation}
		\begin{aligned}
			\tilde{d}_{i,j}^{2}=&\frac{1}{2}\left[\text{\ensuremath{\mathbf{y}}}_{i}-\mathbf{y}_{j}\right]^{T}\mathbf{M}\left({\bf y}_{i}\right)^{\dagger}\left[\text{\ensuremath{\mathbf{y}}}_{i}-\mathbf{y}_{j}\right]+\frac{1}{2}\left[\text{\ensuremath{\mathbf{y}}}_{i}-\mathbf{y}_{j}\right]^{T}\mathbf{M}\left({\bf y}_{j}\right)^{\dagger}\left[\text{\ensuremath{\mathbf{y}}}_{i}-\mathbf{y}_{j}\right] \\ 
			\approx & \left\Vert \text{{\bf x}}_{i}-\text{{\bf x}}_{j}\right\Vert^{2} = d_{i,j}^{2}\label{eq:int_dist_approx}
		\end{aligned}
	\end{equation}
	This approximation has been used is several papers \cite{dsilva2013nonlinear,dsilva2015data,dsilva2015parsimonious,duncan2013identifying,mishne2015graph,singer2008non,talmon2012parametrization,talmon2013empirical,talmon2015intrinsic,talmon2015manifold},
	error analysis for it was presented in \cite{dsilva2015data,singer2008non} and it was found to be:
	\[
	d_{i,j}^{2}=\frac{1}{2}\left[\text{\ensuremath{\mathbf{y}}}_{i}-\mathbf{y}_{j}\right]^{T}\mathbf{M}\left({\bf y}_{i}\right)^{\dagger}\left[\text{\ensuremath{\mathbf{y}}}_{i}-\mathbf{y}_{j}\right]+\frac{1}{2}\left[\text{\ensuremath{\mathbf{y}}}_{i}-\mathbf{y}_{j}\right]^{T}\mathbf{M}\left({\bf y}_{j}\right)^{\dagger}\left[\text{\ensuremath{\mathbf{y}}}_{i}-\mathbf{y}_{j}\right]+\ensuremath{\mathcal{O}}\left(\left\Vert \text{\ensuremath{\mathbf{y}}}_{i}-\mathbf{y}_{j}\right\Vert ^{4}\right)
	\]
	We show the nature of this approximation on an artificial data set in \cref{ssec:simulated_data_Intrinsic_isometric_embedding}. It is evident from both the empirical results from the given example and the rigorous error analysis given in \cite{dsilva2015data} that this approximation is valid only for points of the observed manifold which are close to each other, i.e. when $\left\Vert\text{\ensuremath{\mathbf{y}}}_{i}-\mathbf{y}_{j}\right\Vert ^{4}$ is small with respect to the higher derivatives of the observation function. In order to generate an embedding from the calculated intrinsic inter-point distance estimations we formulate a \ac{LS-MDS} optimization problem using the weighted version of the stress function :
	\begin{equation}
	\sigma\left(\widetilde{\mathbf{X}}\right)=\sum_{i<j}w_{i,j}\left(\left\Vert \mathbf{\widetilde{x}}_{i}-\mathbf{\widetilde{x}}_{j}\right\Vert -\tilde{d_{i,j}}\right)^{2}\label{eq:w-intrinisc_stress}
	\end{equation}
	Where unknown (or unreliable) distances estimations are given zero weights. Determining the exact scale at which we can ``trust'' the approximation \cref{eq:int_dist_approx} is still an open issue but this is a common problem in non-linear dimensionality reduction and dealing with this is out of the scope of this work. We will say that use of between 5 and 15 nearest neighbors around each sample point has proven to be a good rule-of-thumb since this gives good graph connectivity while using mostly reliable distance approximations. We remark that in \cite{singer2008non} this approximation was used in order to create an intrinsic variant of the diffusion maps method. There, the locality of the approximation did not pose a problem due to the exponentially decaying diffusion kernel in which it was used, which made longer distances and their approximation errors insignificant. However, in our work, we aim to retrieve the actual Euclidean geometry of the intrinsic space and it is not immediately clear if this locally approximated intrinsic geometry uncovers enough information about the intrinsic manifold in order to fully recover it. The possibility of retrieving the intrinsic structure only given a subset of known inter-point distances can be reduced to the question of whether a not a inter-point \ac{EDM} originating from points known to reside in an $n$-dimensional Euclidean space can be uniquely completed given only a subset of the distances. This problem is known as the \ac{EDMCP} \cite{alfakih1999solving,fang2012euclidean}. It is known that under cretin conditions it is possible to uniquely determine all missing distances from a subset of of distance, this is usually done using semi-definite programming, which makes it untraceable for problems involving large numbers of points (which are the case of interest in most manifold learning application), however this motivates us to believe that given a number of known distance constrains coupled with the requirement that the points lay in a $n$-dimensional Euclidean space might be enough in order to fully constrain the problem allowing for a unique solution which represents the true latent structure.
	
	\subsection{Intrinsic geodesic distance approximation and intrinsic Isomap}
	\label{ssec:Global-geometry-approximation}
	
	A problem common with \ac{LS-MDS} it is that due to it being a non-convex optimization problem which convergence to a local (as opposed to global) minima. In order to improve the chances of converging to a good final solution one must provide in with a good initial point. In this case this means a good initial embedding into $n$-dimensional space. This initial embedding is then tweaked during the optimization until complete convergence. In order to construct this initial embedding we create an intrinsic variant of the Isomap algorithm.
	
	To do so we harness the ability to calculate local geometry in an intrinsic fashion and use the local distances approximated using \cref{eq:int_dist_approx} in order to approximate all intrinsic geodesic distances directly from the observed manifold. To understand why this is possible we look look at the definition of path lengths:
	\[
	L_{\mathcal{X},h}\left[\gamma\right]=L_{\mathcal{Y},f_{*}}\left[f\left(\gamma\right)\right]=\int_{a}^{b}\sqrt{f_{*f\left(\gamma(t)\right)}\left(f\left(\gamma\left(t\right)\right)',f\left(\gamma\left(t\right)\right)'\right)}dt
	\]	
	This equation states that all length calculations for corresponding paths on the isometric manifolds $\mathcal{X}$ and $\mathcal{Y}$ will be identical if one uses the push-forward metric as the metric on the observed manifold $\mathcal{Y}$. This gives us a way to calculate intrinsic path lengths using only observed data assuming that $\left(\frac{df}{dx}\left({\bf x}_{i}\right)\frac{df}{dx}\left({\bf x}_{i}\right)^{T}\right)^{\dagger}$is known everywhere on the manifold. The push-forward metric accounts for local stretching or contraction of the manifold due to the observation function and changes the way we measure distances in order to compensate for this. This is similar to the role the transformation Jacobian determinant has when performing a change of variables when calculating integrals or transforming probability density functions. Using these intrinsic curve length calculation one can also calculate geodesic distances as the infimum over all paths in $\mathcal{Y}$ which connect $f\left({\bf x}_{i}\right)$ and $f\left({\bf x}_{j}\right)$ where the push-forward metric is used as a metric on the observed manifold. Since only a final set of points sampled from these manifold is given, this minimization over all contentious paths can be approximated by a minimization between all paths passing through a limited set of sample point. This presents a problem which is equivalent to the shortest-path problem where only distances for which we have proper distance approximations are represented by edges on the graph.
	
	Once all intrinsic inter-point geodesic distances are approximated one can use \ac{MDS} to create an $n$-dimensional embedding in which Euclidean distances best respect these distances. This can be seen as an intrinsic variant of the well known Isomap method. The main difference is that instead of assuming local isometry (as is the case in Isomap) we use the push-forward metric instead of the Euclidean metric on the observed manifold in order to explicitly guarantee local isometry. Since geodesic and Euclidean distances are equal to each other for convex spaces, this intrinsic Isomap method gives an intrinsic-isometric embedding for convex intrinsic manifolds as seen in \cref{fig:Intrinsic-isomap-convex}
	
	\begin{figure}[h]
		\begin{centering}
			\begin{minipage}[b][1\totalheight][c]{0.3\columnwidth}%
				\begin{center}
					\includegraphics[width=1\textwidth]{figures/Chapter_3/convex/intrinsic}
					\par\end{center}%
			\end{minipage}\hfill{}%
			\begin{minipage}[b][1\totalheight][c]{0.3\columnwidth}%
				\begin{center}
					\includegraphics[width=1\textwidth]{figures/Chapter_3/convex/observed}
					\par\end{center}%
			\end{minipage}\hfill{}%
			\begin{minipage}[b][1\totalheight][c]{0.3\columnwidth}%
				\begin{center}
					\includegraphics[width=1\textwidth]{figures/Chapter_3/convex/embeddingpng}
					\par\end{center}%
			\end{minipage}
			\par\end{centering}
		
			\begin{centering}
				\begin{minipage}[b]{0.3\columnwidth}%
					\begin{center}
						\includegraphics[width=1\textwidth]{figures/Chapter_3/non_convex/intrinsic}
						\par\end{center}%
				\end{minipage}\hfill{}%
				\begin{minipage}[b]{0.3\columnwidth}%
					\begin{center}
						\includegraphics[width=1\textwidth]{figures/Chapter_3/non_convex/observed}
						\par\end{center}%
				\end{minipage}\hfill{}%
				\begin{minipage}[b]{0.3\columnwidth}%
					\begin{center}
						\includegraphics[width=1\textwidth]{figures/Chapter_3/non_convex/embedding}
						\par\end{center}%
				\end{minipage}
				\par\end{centering}
		\caption{Intrinsic Isomap applied to a convex manifold\label{fig:Intrinsic-isomap-convex}}
	\end{figure}
	
	However this is not true in general and as was discussed in \cite{rosman2010nonlinear} for non-convex manifolds, Euclidean and geodesic distances are not the same and the embedding produced will not respect the intrinsic Euclidean geometry. Indeed on a non-convex data set we see that this method does not recover the structure of the intrinsic space as seen in \cref{fig:punctured_severed_sphere_intrinsic_isomap_embedding}. This shows that simply making the Isomap algorithm intrinsic does not suffice in order to produce an intrinsic-isometric embedding in general however resulting embedding might be ``close-enough'' to an embedding which minimizes the stress function in \cref{eq:w-intrinisc_stress} so that further optimization will converge to a good local minima resulting in an embedding which is intrinsic and isometric.
	
	With this strategy we use a combined approach. Using the embedding generated by our intrinsic variant of Isomap as a initial point for the optimization gives us the advantage of the global optimality of a similar embedding problems, then further optimization of the desired cost function lead to a convergence to a local minima of our exact	problem.This combines the benefits of both eigen-decomposition and non-convex optimization.Eigenvalue decomposition provides a globally optimal embedding given a full inter-point distance matrix however the bedimming generated might be distorted due to to non-convexity of the intrinsic manifold, on the other hand non-convex optimization of the stress function is not effected by non convexity however it will not converge to a good local minimum unless it has a good initial embedding. 
	
	\subsection{Stress minimization and final embedding}
	\label{ssec:Embedding-correction-for}
	
	Once an initial embedding is built using our intrinsic variant of the Isomap method, further minimization of stress function is performed iteratively with \ac{SMACOF}.
	
	It is important to remark that our initial goal, as detailed in \cref{sec:setting}, was to minimize the full, unweighted stress function with respect to the true intrinsic Euclidean distances. Instead we then minimize a weighted stress function where only short intrinsic distances (which we were able to approximate from the observed data) appear. This is similar to common setting in machine learning where a partially known and approximated cost function is minimized with the hope that the true cost function, corresponding in our case to the full and exact stress, in also minimized. It seem that we might suffer from over-fitting if the problem is not constrained enough by the existing distance and low-rank restrictions on the embedding. To see if this is the case we plot the observed partial stress function along the full true stress function. As can be seen in \cref{fig:Stress-Plot-Full} both these function exhibit similar behaviors during the optimization process.
	\begin{figure}[h]
		\begin{centering}
			\includegraphics[width=0.7\textwidth]{figures/Chapter_3/stress_real_vs_observed}
			\par\end{centering}
		\caption{Observed partial stress and true full stress during \ac{LS-MDS} optimization\label{fig:Stress-Plot-Full}}
	\end{figure}
	At first it might seem surprising that the real stress is lower then the observed stress estimated intrinsic Euclidean distances. To explain this we remember that the intrinsic distance estimations used for the cost function are only noisy approximation of the true intrinsic distances and as such might disagree with each other and might not necessarily be exactly embeddable into a low-dimensional space. As a result a certain amount of stress will exist in the embedding with respect to the noisy distance approximations. The \ac{LS-MDS} process aggregates these noisy intrinsic distance estimations into a single embedding, this has a denoising effect which reduces the stress of the final embedding with respect to the true distances.
	
	\subsection{Multi-scale scheme}
	
	When the non-convexity of the intrinsic manifold is not very significant, the initial embedding received using our intrinsic variant of the Isomap method (described in \cref{ssec:Global-geometry-approximation}) is, in many cases, close enough to the actual embedding as to allow convergence of the weighted \ac{LS-MDS} minimization to a good solution as described in \cref{ssec:Embedding-correction-for}. However, when the non-convexity of the intrinsic manifold i significant, this initial embedding might not be close enough to the actual intrinsic structure and the \ac{LS-MDS} optimization will not converge to a quality local minimum. To avoid this situation we suggest a multi-scale optimization scheme.
	
	We start from a small scale on the observed manifold, applying this method to small patches of the manifold which are convex or approximately-convex, thus improving the chance of the \ac{LS-MDS} optimization to converge to a good solution. Assuming that the algorithm works properly on such a small scale and a good embedding into Euclidean space is received, a straightforward calculation of approximate intrinsic Euclidean distances from the embedding itself is possible and we approximate all inter-point distances in this patch via simple Euclidean distance measurement in the received embedding. This provides an approximation for the previously unknown or missing distances. This approximation does not relay on convexity of the intrinsic manifold and avoids the distortions induced by non-convexity when these distances were previously approximated using shortest-path algorithm. With intrinsic distances now approximated on a larger scale than before we can now increase the scale on which the algorithm operates, again allowing for the introduction of some additional slight non-convexity. At the subsequent iterations of the algorithm, the scale is increased iteratively and the non-convexity induces by the new points is again tackled in the same way. This process is repeated until an embedding of all points into Euclidean space is attained which respect the Euclidean intrinsic distance. The definition of operation scales and their increment can be easily done using the intrinsic geodesic distances which were already approximated as described in \cref{ssec:Global-geometry-approximation}. This gradual increase in scale is visualized in \cref{fig:Increasing-operation-scales}. We remark that although the visualization is in the intrinsic space, the estimation of the geodesic distances and hence the scale selection is performed directly on the observed manifold.
	
	\begin{figure}[h]
		\begin{minipage}[t]{0.19\columnwidth}%
			\begin{center}
				\includegraphics[width=1\textwidth]{figures/Chapter_3/scale_1}
				\par\end{center}%
		\end{minipage}\hfill{}%
		\begin{minipage}[t]{0.19\columnwidth}%
			\begin{center}
				\includegraphics[width=1\textwidth]{figures/Chapter_3/scale_2}
				\par\end{center}%
		\end{minipage}\hfill{}%
		\begin{minipage}[t]{0.19\columnwidth}%
			\begin{center}
				\includegraphics[width=1\textwidth]{figures/Chapter_3/scale_3}
				\par\end{center}%
		\end{minipage}\hfill{}%
		\begin{minipage}[t]{0.19\columnwidth}%
			\begin{center}
				\includegraphics[width=1\textwidth]{figures/Chapter_3/scale_4}
				\par\end{center}%
		\end{minipage}\hfill{}%
		\begin{minipage}[t]{0.19\columnwidth}%
			\begin{center}
				\includegraphics[width=1\textwidth]{figures/Chapter_3/scale_5}
				\par\end{center}%
		\end{minipage}
		
		\caption[Increasing operation scales]{Increasing operation scales - 200 points are operated on on the first
			scale and an additional 200 points are added at each scale \label{fig:Increasing-operation-scales}}
	\end{figure}
	
	A pitfall that might occur in this process is that the increase of the scale of the algorithm is too large resulting in introduction of major non-convexity and poor distance estimation, luckily we can monitor the process of scale increase by calculating the observed partial stress of the generated embedding with respect to the approximated short-intrinsic distances and different scales can be attempted till a good stress value is reached.
	
	Another way in which this process can be made more robust is by starting this multi-scale schema form multiple different starting points and then choosing the embedding which results in the least amount of partial observed stress. The selection of stating point can be done by randomly selecting sample points or preferably by clustering the data using the already approximated inter-point intrinsic geodesic distances, which leads to starting points that are distributed across the manifold and can lead to different final results.
	
	\begin{figure}[h]
		\begin{centering}
			\includegraphics[width=0.7\textwidth]{figures/Chapter_3/center_point_clustering}
			\par\end{centering}
		\caption[Clustering and starting point selection]{Clustering and starting point selection using approximated intrinsic
			geodesic distances. Different colors represent different clusters.
			The bigger red points represent the cluster centers and will be used
			as different starting points of the multi-scale schema\label{fig:Clustering-by-geodesic}}
	\end{figure}

	\externaldocument{paper.tex}


	\section{Intrinsic Metric Estimation}
		\label{sec:Intrinsic-Metric-Estimation}

	In the algorithm presented in \cref{sec:Intrinsic-isometric-manifold-learning}
	we assumed that the matrix $\mathbf{M}\left({\bf y}_{i}\right)$ was
	given for all $\text{\ensuremath{\mathbf{y}}}_{i}\in\mathcal{Y}_{s}$.
	This matrix was used for the computation of an intrinsic metric on
	the observed manifold $\mathcal{Y}$ that locally ``cancels-out''
	the distorting effect of the observation function. The use of this
	intrinsic metric allowed us to approximate intrinsic Euclidean and
	geodesic inter-point distances and was one of the key ingredients
	to uncovering the intrinsic geometrical structure of the latent manifold.
	The ability to apply the algorithm suggested in \cref{sec:Intrinsic-isometric-manifold-learning}
	to practical real-life problems therefore depends on how reasonable
	it is to assume that $\mathbf{M}\left({\bf y}_{i}\right)$ is known
	or can be robustly approximated from observed data.
	
	In this chapter, we address this issue by first giving a few examples
	of settings under which we can, under different prior assumptions,
	approximate $\mathbf{M}\left({\bf y}_{i}\right)$ from approximately-linear
	patches of the observe manifold. We then recognize that due to non-linear
	nature of the observation function, estimation methods requiring approximate
	linearity are local in nature and thus not robust and susceptible
	to noise when data is not sufficiently densely sampled. To overcome
	this, we propose a non-local metric estimation method where we use
	an \ac{ANN} parametric regressor for $\mathbf{M}\left({\bf y}_{i}\right)$
	on the whole observed manifold simultaneously. We show that by using
	smooth non-linearities in the net and additionally restricting the
	net structure and its weights, we provide sufficient regularization
	to the estimation process, making it more robust to the factors mentioned
	above. Lastly, we show that improving the intrinsic metric estimation
	makes the algorithm presented in \cref{sec:Intrinsic-isometric-manifold-learning}
	applicable to problems where other local metric estimation methods
	are not successful, extending the range of problems to which we can
	apply our method of intrinsic-isometric manifold learning.
	
	

	\subsection{Global metric estimation}
	\label{ssec:Global-metric-estimation}
	
	A common characteristic of the estimation methods described in the
	Section \cref{sec:Local-intrinsic-metric} is that they operated on
	a scale on which the observation function $f$ is approximately linear
	and can be well approximated by its Jacobian. Working on larger scales,
	on which this assumption is not true, and taking samples which are
	further away from the point of estimation will have a negative effect
	on the resulting estimation since distant samples experience different
	observation function Jacobians and therefore have different intrinsic
	metrics. The curvature of the manifold dictates how fast the Jacobian
	and the metric changes along the manifold and therefore restricts
	the scale at which samples are relevant for estimation of the metric
	at a specific point. High non-linearly of observation function can
	make the scale on which the estimation is performed very small. If
	the data is not sufficiently densely sampled (as is often the case
	in practice in realistic situations), the number of data points that
	will be incorporated into the estimation of a single local metric
	will be small, resulting in an estimation which has high variance
	and susceptible to observation noise. This non-robustness is not unique
	to these specific metric estimation methods and many non-linear estimation
	problems and manifold learning techniques are non-robust due to their
	local nature as discussed in \cite{bengio2004non}. 
	
	A possible way to overcome this is by imposing a global regularization
	factor on the metric estimation. Such a term should encourage metric
	estimations which are more ``reasonable'', not only on a local scale
	but also globally. In our setting we assume that the observation function
	$f$ is a continuously-differentiable and smooth, therefore the intrinsic
	metric (which we showed depended only on the observation function
	Jacobian) should also change smoothly on the manifold, making estimations
	such at the one received for the sparsely sampled manifold in \cref{fig:Intrinsic-metric-estimation} very unlikely. Adding a restriction
	that metrics estimated at close points on the observed manifold will
	be similar to each other should steer the estimation towards more
	``reasonable'' outcomes and reduces the estimator variance making
	it more robust.

	\subsection{Maximum-Likelihood intrinsic metric estimation\label{sec:Intrinsic-metric-estimation}}
	
	In the setting presented in Subsection \cref{ssec:Intrinsic-isotropic-GMM},
	we saw that the observed data can be regarded as a \ac{GMM} model
	with covariance matrices which are effected by the observation function
	Jacobian, as seen in Equation \cref{eq:prob-pca}. An unconstrained
	maximum-likelihood estimation of $\mathbf{M}\left({\bf y}_{i}\right)$
	for that setting produces a local estimator which only used data from
	a single Gaussian and did not impose any smoothness on the metrics
	estimated on the whole manifold. As a result, this estimation suffers
	from the problems described in \cref{subsec:Benefits-of-non-local}.
	To overcome this we suggest adopting a similar probabilistic viewpoint
	but to instead use a constrained non-local maximum-likelihood estimator.
	The intrinsic isotropic \ac{GMM} model described in \cref{ssec:Intrinsic-isotropic-GMM}
	results in the following approximate global log-likelihood function
	for the observed data:
	\begin{equation}
	\mathcal{L}\left(\left\{ f\left(\mathbf{x}_{i}\right)\right\} _{i=1}^{N},\left\{ \frac{df}{dx}\left(\mathbf{x}_{i}\right)\frac{df}{dx}\left(\mathbf{x}_{i}\right)^{T}\right\} _{i=1}^{N}\right)=-\frac{1}{2}\sum_{i=1}^{N}N_{i}\left\{ d\text{ln}\left(2\pi\right)+\text{ln}\left|\mathbf{C}_{i}\right|+\text{Tr}\left(\mathbf{C}_{i}^{-1}\mathbf{S}_{i}\right)\right\} 
	\end{equation}
	Where, $N$ is the number of Gaussian distribution in the \ac{GMM},
	$N_{i}$ is the number of points sampled from the $i$-th Gaussian
	distribution, $\mathbf{S}_{i}$ is the sample covariance of the $i$-th
	observed Gaussian distribution and $\mathbf{C}_{i}$ denotes the covariance
	which can be explained by the model:
	\begin{equation}
	\mathbf{C}_{i}=\sigma_{int}^{2}\frac{df}{dx}\left(\mathbf{x}_{i}\right)\frac{df}{dx}\left(\mathbf{x}_{i}\right)^{T}+\sigma_{obs}^{2}\mathbf{I}=\sigma_{int}^{2}\mathbf{M}\left({\bf y}_{i}\right)+\sigma_{obs}^{2}\mathbf{I}
	\end{equation}
	This log-likelihood is also presented in \cite{tipping1999mixtures}.
	If we are free to estimate $\mathbf{M}\left({\bf y}_{i}\right)$ by
	choosing its most likely value for each cluster independently, we
	get the same metric estimation described in \cref{ssec:Intrinsic-isotropic-GMM}.
	In order to avoid this, we instead assume that the observation function
	Jacobian belongs to a parametric family of functions which we denote
	$\mathbf{J}\left({\bf y}_{i}|\theta\right):\mathbb{R}^{m}\to\mathbb{R}^{m\times n}$
	where $\theta$ represents the parameterization of this family. This
	family of matrix valued functions will represent the unknown observation
	function Jacobian $\frac{df}{dx}\left(f^{-1}\left({\bf y}_{i}\right)\right)$
	. Estimation is performed by choosing a specific function from this
	family which best explains the observed data. Mathematically, this
	is equivalent to choosing $\theta$ that maximizes the parameterized
	log-likelihood function:
	\begin{equation}
	\mathcal{L}\left(\left\{ {\bf y}_{i.j}\right\} |\theta\right)=-\frac{1}{2}\sum_{i=1}^{N}N_{i}\left\{ d\text{ln}\left(2\pi\right)+\text{ln}\left|{\bf C}\left({\bf y}_{i}|\theta\right)\right|+\text{Tr}\left(\mathbf{\mathbf{C}}\left({\bf y}_{i}|\theta\right)^{-1}\mathbf{S}_{i}\right)\right\} \label{eq:Ann-cost}
	\end{equation}
	Where:
	\[
	\mathbf{C}\left({\bf y}_{i}|\theta\right)=\sigma_{int}^{2}\mathbf{J}\left({\bf y}_{i}|\theta\right)\mathbf{J}\left({\bf y}_{i}|\theta\right)^{T}+\sigma_{obs}^{2}\mathbf{I}
	\]
	The restriction of the Jacobian function to a specific class of possible
	functions, limits the possible estimation results. By using a parametric
	family of smooth functions we can avoid unreasonable estimations results
	corresponding to rapidly varying metrics on the manifold, thus making
	the estimation more robust. Finding the optimal function from the
	family estimates the Jacobian (and the intrinsic metric) for all points
	in the observed space simultaneously, unlike in the unconstrained
	case where an optimization is performed multiple times ,once for each
	cluster. By representing the observation function Jacobian as the
	output of a single function we tie to estimations of the intrinsic
	metrics on the whole manifold manifold . This estimation is no longer
	local and samples over the whole manifold can potentiality effect
	the estimation on other points on the manifold. A estimation for a
	local metric might fit the observed data perfectly at one point but
	if the estimated Jacobian $\mathbf{J}\left(\mathbf{y}|\theta\right)$
	does not provide a reasonable explanation for the observed data on
	the whole manifold it will be deemed unlikely.

	The log-likelihood in \cref{eq:Ann-cost} only differs by a constant from the following function:
	\begin{equation}
	\begin{aligned}R\left(\left\{ {\bf y}_{i.j}\right\} |\theta\right) & =-\sum_{i=1}^{N}N_{i}D_{KL}\left(\mathbf{C}_{i}||\mathbf{S}_{i}\right)\\
	& =-\sum_{i=1}^{N}N_{i}\left\{ \text{ln}\frac{\left|\mathbf{C}_{i}\right|}{\left|\mathbf{S}_{i}\right|}+\text{Tr}\left(\mathbf{C}_{i}^{-1}\mathbf{S}_{i}\right)\right\} \\
	& =\mathcal{L}\left(\left\{ {\bf y}_{i.j}\right\} |\theta\right)+\frac{1}{2}\sum_{i=1}^{N}N_{i}\left\{ d\text{ln}\left(2\pi\right)-\text{ln}\left|\mathbf{S}_{i}\right|\right\} \\
	& =\mathcal{L}\left(\left\{ {\bf y}_{i.j}\right\} |\theta\right)+constant
	\end{aligned}
	\label{eq:KL}
	\end{equation}
	Where $D_{KL}\left(\mathbf{C}_{i}||\mathbf{S}_{i}\right)$ is the well known Kullback\textendash Leibler divergence between Gaussian distributions with the same mean and two different covariances matrices $\text{\ensuremath{\mathbf{C}_{i}}}$ and $\mathbf{S}_{i}$. This divergence receives a minimal value of zero when these covariances are equal. Thus maximization of the likelihood is equivalent to minimization of the discrepancy, in terms of the Kullback\textendash Leibler divergence, between the covariance that can be explained by the estimation of the Jacobian function and the sample covariance observed in the data cluster. 
	
	\subsection{Estimation and regularization using artificial neural networks}
	
	\cref{sec:Intrinsic-metric-estimation} gave us a probabilistic framework for non-local estimation of the intrinsic metric by using a parametric family of functions, however it did not specify which parametric family to use. We suggest the use of an \ac{ANN} for this purpose. Neural networks are compositions of linear transformations and non-linear operation nodes which can be adapted to approximate a specific function by choosing the values of the linear transformations in the network, these are also called the ``weights'' of the network. Thus, the parameter vector $\theta$ to be optimized represents, for the case of \ac{ANN}, all the weights of the network. By choosing the non-linear operation nodes to be smooth functions, we can explicitly impose the smoothness of net output function. We used the commonly used standard sigmoid function:
	
	\[
	s\left(x\right)=\frac{1}{1+e^{-x}}
	\]
	
	The inputs to the network are coordinates of points in the observed measurement space and the outputs are vectors of dimension $n\cdot m$ which are reshaped to an $n\times m$ matrix $\mathbf{J}\left({\bf y}_{i}|\theta\right)$ and serve as approximations of the observation function Jacobian at the corresponding input points. The optimal estimation is achieved by optimizing over the weights of the network so that the likelihood function \cref{eq:Ann-cost} is maximized. The neural network structure used in this work contains two hidden layer and an additional linear layer at the output, which was added in order to allow for automatic scaling of the output to the correct problem scale. Our suggested network structure is visualized in \cref{fig:Suggested--structure}.
	
	\begin{figure}[h]
		\begin{centering}
			\begin{minipage}[t]{1\columnwidth}%
				\begin{center}
					\graphicspath{ {figures/Chapter_4/} } 
					\def\layersep{2.8 cm} 
					\def\nodeSize{40 pt}
					\def\nInputNodes{3} 
					\def\nInputNodesMinusOne{2} 
					\def\nInputNodesPlusOne{4}
					\def\nHiddenNodes{5} 
					\def\nHiddenNodesMinusOne{4} 
					\def\nHiddenNodesPlusOne{6} 
					
					\begin{tikzpicture}[shorten >=1pt,->,draw=black!50, node distance=\layersep,thick,scale=0.9, every node/.style={transform shape}] 
					\tikzstyle{every pin edge}=[<-,shorten <=1pt]     
					\tikzstyle{neuron}=[circle,draw, ultra thick,fill=black!25,minimum size=27pt,inner sep=0pt]     	
					\tikzstyle{input neuron}=[neuron, fill=green!50]     
					\tikzstyle{output neuron}=[neuron, fill=red!50]     
					\tikzstyle{hidden neuron}=[neuron, fill=yellow!50]     
					\tikzstyle{encoding neuron}=[neuron, fill=gray!50]    
					\tikzstyle{annot} = [text width=4em, text centered]
					
					% Draw the input layer nodes     
					\path[yshift=-\nodeSize] 		
					node[input neuron, pin=left:$y_1$] (I-1) at ({0*\layersep},-1*\nodeSize) {};
					\path[yshift=-\nodeSize] 		
					node[input neuron, pin=left:$y_2$] (I-2) at ({0*\layersep},-2*\nodeSize) {};
					\path[yshift=-\nodeSize] 		
					node[input neuron, pin=left:$y_m$] (I-\nInputNodes) at (0*\layersep,-\nInputNodesPlusOne*\nodeSize) {};
					\path[-] (I-\nInputNodesMinusOne) edge (I-\nInputNodes) [dotted];
					
					% Draw the first hidden layer nodes     
					\foreach \name / \y in {1,...,\nHiddenNodesMinusOne}         
					\path[yshift=0.0cm] 	    	
					node[hidden neuron] (H1-\name) at (1*\layersep,-\y*\nodeSize) {{\includegraphics[width=25pt]{nodeNonLin.png}}}; 		
					\path[yshift=-\nodeSize]         
					node[hidden neuron] (H1-\nHiddenNodes) at (1*\layersep,-\nHiddenNodes*\nodeSize) {{\includegraphics[width=25pt]{nodeNonLin.png}}};
					\path[-] (H1-\nHiddenNodesMinusOne) edge (H1-\nHiddenNodes) [dotted];
					
					% Draw the second hidden layer nodes     
					\foreach \name / \y in {1,...,\nHiddenNodesMinusOne}         
					\path[yshift=0.0cm]             
					node[hidden neuron] (H2-\name) at (2*\layersep,-\y*\nodeSize) {{\includegraphics[width=25pt]{nodeNonLin.png}}};
					\path[yshift=-\nodeSize]
					node[hidden neuron] (H2-\nHiddenNodes) at (2*\layersep,-\nHiddenNodes*\nodeSize){{\includegraphics[width=25pt]{nodeNonLin.png}}}; 	
					\path[-] (H2-\nHiddenNodesMinusOne) edge (H2-\nHiddenNodes) [dotted]; 
					
					% Draw the output layer node     
					\path[yshift=-\nodeSize]             
					node[output neuron, pin={[pin edge={black, solid, ->}, black,pin distance={0.8*\nodeSize}]right:${\mathbf{J}\left({\bf y}|\theta\right)}_{1,1}$}] (O-1) at (3*\layersep,-1*\nodeSize) {};
					\path[yshift=-\nodeSize]             
					node[output neuron, pin={[pin edge={black, solid, ->}, black,pin distance={0.8*\nodeSize}]right:${\mathbf{J}\left({\bf y}|\theta\right)}_{1,2}$}] (O-2) at (3*\layersep,-2*\nodeSize) {};
					\path[yshift=-\nodeSize]             
					node[output neuron, pin={[pin edge={black, solid, ->}, black,pin distance={0.8*\nodeSize}]right:${\mathbf{J}\left({\bf y}|\theta\right)}_{m,n}$}] (O-\nInputNodes) at (3*\layersep,-\nInputNodesPlusOne*\nodeSize) {};
					\path[-] (O-\nInputNodesMinusOne) edge (O-\nInputNodes) [dotted];
					
					%%% Connect N     
					\foreach \source in {1,...,\nInputNodes}         
					\foreach \dest in {1,...,\nHiddenNodes}             
					\path (I-\source) edge (H1-\dest);
					
					\foreach \source in {1,...,\nHiddenNodes}         
					\foreach \dest in {1,...,\nHiddenNodes}             
					\path (H1-\source) edge (H2-\dest);                
					
					\foreach \source in {1,...,\nHiddenNodes}         
					\foreach \dest in {1,...,\nInputNodes}             
					\path (H2-\source) edge (O-\dest);
					
					\node[annot,above of=H1-1, node distance=2cm] {Hidden Layer 1}; 
					\node[annot,above of=H2-1, node distance=2cm] {Hidden Layer 2};
					\node[annot,above of=I-1, node distance=2cm] {Input Layer};
					\node[annot,above of=O-1, node distance=2cm] {Output Layer}; 
					
					\coordinate (Dash-11) at ({0*\layersep+0.8*\nodeSize},0) {};
					\coordinate (Dash-12) at ({2*\layersep+0.8*\nodeSize},0) {};
					\coordinate (Dash-13) at ({3*\layersep+0.8*\nodeSize},0) {};
					
					\end{tikzpicture} 
					\par\end{center}%
			\end{minipage}
			\par\end{centering}
		\caption{Suggested \ac{ANN} structure \label{fig:Suggested--structure}}
	\end{figure}
	\acp{ANN} also have a few additional advantages over other smooth parametric function families. They are very general and have proven to perform well in a very wide spectrum of application, their construction and optimization is well researched, many methods exist to regularize them making it easier to explicitly and implicitly impose smoothness of the result and finally, due to their extreme popularity there are many software and hardware solutions allowing for efficient and fast optimization of these networks. 
	
	\subsection{Hyper-parameter selection}
		\label{ssec:Hyper-parameter-selection-and}
		
	In addition to using smooth non-linearities in order to impose smoothness of the estimated metric, one can also control the estimated function complexity by limiting the number of non-linear nodes in the network and by using a weight decay term which encourages the use of smaller weights in the network, preventing the output function from varying too much with changes in the input \cite{krogh1991simple}. Choosing these network hyper-parameters is an important part of the use of neural networks. Restricting the network too much limits its ability to describe the complexity of the observation function by overly smoothing the metric variation, on the other hand, using a network with too many degree of freedom would cause the estimation to be unrestricted and would lead to and effectively local and non-robust estimator.
	
	Since we are working in a unsupervised setting with no ground truth value for the intrinsic metric anywhere on the manifold it is not immediately clear how one can choose these hyper-parameters for the network. We employed two approaches which yielded similar results. 
	
	The first is through a validation set, where we set aside a number of measurements which are not used to the training of the network but are used to estimate the true log-likelihood of the learned estimator. An over-fitted or under-constrained model would be able to explain
	fit itself to the observed data but it will likely not be able to properly explain the validation set due to its high variance. In our tests we used common regular k-fold cross-validation where $20\%$ of the observed samples were used for validation purposes, this was done 20 time for each set of hyper parameters, where the validation set was sampled randomly from the whole data at each iteration.
	
	In the following example we sample the data with a \ac{GMM} of $N=500$ cluster with $N_{i}=5$ samples generated from each cluster. Out of the $N$ clusters, $N_{Valid}=100$ clusters are used for validation and $N_{Train}=400$ clusters are used for training. We plot the error graphs of the training set the validation set and the ground truth error, by using the real analytical Jacobians for different net sizes and weight decays, we perform a number of these training and average the results on the validation set. 
	
	We wish to choose the optimal number of nodes for each of the two layers of the network, we examine the possibilities of $10,20,50$
	nodes and also different magnitudes of the weight decay term. The results of training for a number of sets of hyper parameter are presented for the Fishbowl data in \cref{fig:learning_curves}. We see that the validation log-likelihood, when behaves very similarly to the log-likelihood of the the real data. This allows us to use the validation log-likelihood as a way to estimate the true performance
	
	\begin{figure}[h]
		\begin{centering}  
			\begin{subfigure}[b]{0.28\linewidth}
				\includegraphics[width=1\textwidth]{figures/Chapter_4/validation/n_10_w_0.0002.png}
				\caption{N=10, w=0.0002}
			\end{subfigure} \hfill
			\begin{subfigure}[b]{0.28\linewidth}
				\includegraphics[width=1\textwidth]{figures/Chapter_4/validation/n_10_w_0.00005.png}
				\caption{N=10, w=0.00005}
			\end{subfigure} \hfill 
			\begin{subfigure}[b]{0.28\linewidth}
				\includegraphics[width=1\textwidth]{figures/Chapter_4/validation/n_10_w_0.00001.png}
				\caption{N=10, w=0.00001}
			\end{subfigure}
		\end{centering}
		\begin{centering}  
			\begin{subfigure}[b]{0.28\linewidth}
				\includegraphics[width=1\textwidth]{figures/Chapter_4/validation/n_20_w_0.0002.png}
				\caption{N=20, w=0.0002}
			\end{subfigure} \hfill
			\begin{subfigure}[b]{0.28\linewidth}
				\includegraphics[width=1\textwidth]{figures/Chapter_4/validation/n_20_w_0.00005.png}
				\caption{N=20, w=0.00005}
			\end{subfigure} \hfill 
			\begin{subfigure}[b]{0.28\linewidth}
				\includegraphics[width=1\textwidth]{figures/Chapter_4/validation/n_20_w_0.00001.png}
				\caption{N=20, w=0.00001}
			\end{subfigure}
		\end{centering}
		\begin{centering}  
			\begin{subfigure}[b]{0.28\linewidth}
				\includegraphics[width=1\textwidth]{figures/Chapter_4/validation/n_50_w_0.0007.png}
				\caption{N=50, w=0.0007}
			\end{subfigure}\hfill
			\begin{subfigure}[b]{0.28\linewidth}
				\includegraphics[width=1\textwidth]{figures/Chapter_4/validation/n_50_w_0.0005.png}
				\caption{N=50, w=0.0005}
			\end{subfigure} \hfill 
			\begin{subfigure}[b]{0.28\linewidth}
				\includegraphics[width=1\textwidth]{figures/Chapter_4/validation/n_50_w_0.0002.png}
				\caption{N=50, w=0.0002}
			\end{subfigure}
		\end{centering}
		\caption{Learning curves for different hyper-parameters \label{fig:learning_curves} }   
	\end{figure}	
	
	A second method that can be used in this case is choosing the hyper-parameter which result in the final embedding with the minimal observed stress with respect to the estimated distances. If metric estimation are poor, due to noise or over-fitting, it is unlikely that the resulting estimated inter-point distances will allow for an embedding in low-dimensional space . Choosing the hyper-parameter which results in distances which ``fit together'' the best is a good indication that it represents an embedding that is close to the true structure. Since the metric estimations are used for distance estimations which are then used in order to embed the point in a low-dimensional space. Errors in the metric estimation will lead to sets of intrinsic distance that do not fit well in low dimensional space. Intuitively it is reasonable that wrongly estimating the distances and metrics will not by chance lead to distances describing a different possibly low-dimensional conformation. In practice we use different value of regularization and choose the option which results in the lowest normalized stress value with respect to the approximated intrinsic distances.
	
	\subsection{Implementation}
		\label{ssec:Net_implementation}
		We implemented the described metric estimation network in Python using Theano \cite{bergstra2010theano} for symbolic derivation of the cost term presented in (\cref{eq:Ann-cost}) with respect to the network weights. To facilitate better and faster optimization the stochastic gradient descent method \ac{ADAM} \cite{kingma2014adam} was used. This method uses momentum and moment estimation over time to automatically and adaptability scale the learning rate of the optimization process for faster convergence. The optimal net hyper-parameters were selected by k-fold cross validation as described in Subsection \cref{ssec:Hyper-parameter-selection-and}.
	
	\externaldocument{paper.tex}
\section{Experimental results} \label{sec:results}

	\subsection{Simulated data}
	
	Consider an intrinsic manifold $\mathcal{X}$ corresponding to a 2-dimensions square with a cross shaped hole in its center which makes it non-convex. This latent manifold is observed via the following observation function:
	\begin{equation} \label{eq:simulated_data_observation_function}
	\mathbf{y\left(\mathbf{x}\right)}=f\left(\mathbf{x}\right)=\text{\ensuremath{\left[\begin{array}{c}
			y_{1}\left(\mathbf{x}\right)\\
			y_{2}\left(\mathbf{x}\right)\\
			y_{3}\left(\mathbf{x}\right) 
			\end{array}\right]}=}
		\left[\begin{array}{c}
	\sin \left(2.5 \cdot x_1 \right)\cdot \sin \left( x_2 \right)\\
	\sin \left(2.5 \cdot x_1 \right)\cdot \cos \left( x_2 \right)\\
	-\sin \left( x_2 \right)
	\end{array}\right]
	\end{equation}
	The observation function embeds the data in 3-dimensional Euclidean space by partially wrapping it on the unit sphere. The resulting observed manifold $\mathcal{Y}=\left\{ f\left(\mathbf{x}\right) \mid \mathbf{x}\in\mathcal{X}\right\}$ has the shape of a severed ball with a cross shaped hole in it. In order to generate the sample subsets $\mathcal{X}_{s}$ and $\mathcal{Y}_{s}$, $N=1000$ points are sampled uniformly with respect to $\mathcal{X}$. 
	
	\subsubsection{Intrinsic isometric embedding} \label{ssec:simulated_data_Intrinsic_isometric_embedding}
	First we evaluate the validity of the intrinsic Euclidean distance estimation presented in \cref{eq:int_dist_approx} and the ability of the embedding method suggested in \cref{sec:Intrinsic-isometric-manifold-learning} to recover the latent geometric structure of the data, given the intrinsic metric over the observed manifold. To do so we analytically calculate $\frac{df}{dx}\left(\mathbf{x}_{i}\right)\frac{df}{dx}\left(\mathbf{x}_{i}\right)^{T}$ by taking the derivative of the observation function given in \cref{eq:simulated_data_observation_function} with respect to the intrinsic latent variable. 
	
	The results of applying the suggested intrinsic isometric embedding algorithm on this data set are displayed in \cref{fig:punctured_severed_sphere}. In \cref{fig:punctured_severed_sphere_intrinsic} we present the intrinsic latent manifold in the intrinsic space and in \cref{fig:punctured_severed_sphere_observed} we present the observed manifold embedded in the observation space. In \cref{fig:punctured_severed_sphere_intrinsic_metric} we visualize the exact intrinsic metric (the push-forward metric) by plotting corresponding ellipses at several sample points. These  ellipses represent the images of circles of equal radius in the intrinsic space according to the estimated metric, this visualizes the amount of local stretch and contraction in each direction that the observed manifold experiences with respect to the latent intrinsic manifold . In \cref{fig:punctured_severed_sphere_intrinsic_dist_est} we plot the ground truth intrinsic Euclidean inter-point distances against the approximated inter-point distance using \cref{eq:int_dist_approx}, this results in a scatter plot where the closer a point is to the diagonal red line, the better the distance estimation of the corresponding intrinsic distance between the point pair is. In \cref{fig:punctured_severed_sphere_intrinsic_dist_est_knn} we plot the same distance estimation scatter plot but restrict the point pairs represented to distances to the selected $k$ nearest neighbors, which are the only distances which are taken into account by our suggested algorithm. For the following results we used $k=30$. 
	
	In order to compare different manifold learning methods we plot for each method the resulting embedding and a scatter plot which for each point pair, compares the Euclidean distance in the resulting embedding to the true intrinsic Euclidean distance as measured in the intrinsic space. In general a concentration of point along the diagonal line represents a better embedding. In order to provide a single quantitative measure of the quality of the reconstruction , we also calculate for each embedding the Stress (\cref{eq:stress}) of the embedding with respect to the true intrinsic structure. These to graphs are plotted for standard Isomap (\cref{fig:punctured_severed_sphere_standard_isomap_embedding} and \cref{fig:punctured_severed_sphere_standard_isomap_stress}, intrinsic Isomap (\cref{fig:punctured_severed_sphere_standard_isomap_embedding} and \cref{fig:punctured_severed_sphere_intrinsic_isomap_stress}) and finally for our suggested intrinsic-isometric algorithm (\cref{fig:punctured_severed_sphere_intrinsic_isometric_embedding} and \cref{fig:punctured_severed_sphere_intrinsic_isometric_stress}).
	
	Similar results on several other synthetic data sets are provided as supplementary material \cref{ssec:Additional-results-manifold-learning}.
	
	\begin{figure}[h]	
		\begin{centering}
			\begin{subfigure}[b]{0.3\linewidth}
				\includegraphics[width=1\linewidth]{figures/Chapter_3/cross_punctured_2d_square_severed_sphere/intrinsic}
				\caption{\label{fig:punctured_severed_sphere_intrinsic}}
			\end{subfigure}\hfill
			\begin{subfigure}[b]{0.3\linewidth}
				\includegraphics[width=1\linewidth]{figures/Chapter_3/cross_punctured_2d_square_severed_sphere/observed}
				\caption{\label{fig:punctured_severed_sphere_observed}}
			\end{subfigure}\hfill
			\begin{subfigure}[b]{0.3\linewidth}
				\includegraphics[width=1\linewidth]{figures/Chapter_3/cross_punctured_2d_square_severed_sphere/metric_local_dense}
				\caption{\label{fig:punctured_severed_sphere_intrinsic_metric}}
			\end{subfigure}
		\end{centering}
		\begin{centering}
			\begin{subfigure}[b]{0.45\linewidth}
				\includegraphics[width=1\linewidth]{figures/Chapter_3/cross_punctured_2d_square_severed_sphere/dist_local_dense}
				\caption{\label{fig:punctured_severed_sphere_intrinsic_dist_est}}
			\end{subfigure}
			\hfill
			\begin{subfigure}[b]{0.45\linewidth}
				\includegraphics[width=1\linewidth]{figures/Chapter_3/cross_punctured_2d_square_severed_sphere/dist_local_dense_knn}
				\caption{\label{fig:punctured_severed_sphere_intrinsic_dist_est_knn}}
			\end{subfigure}
		\end{centering}
		\begin{centering}
			\begin{subfigure}[b]{0.32\linewidth}
				\includegraphics[width=1\linewidth]{figures/Chapter_3/cross_punctured_2d_square_severed_sphere/standard_isomap_embedding}
				\caption{\label{fig:punctured_severed_sphere_standard_isomap_embedding}}
			\end{subfigure} \hfill
			\begin{subfigure}[b]{0.32\linewidth}
				\includegraphics[width=1\linewidth]{figures/Chapter_3/cross_punctured_2d_square_severed_sphere/embedding_intrinsic_isomap_local_dense}
				\caption{\label{fig:punctured_severed_sphere_intrinsic_isomap_embedding}}
			\end{subfigure} \hfill
			\begin{subfigure}[b]{0.32\linewidth}
				\includegraphics[width=1\linewidth]{figures/Chapter_3/cross_punctured_2d_square_severed_sphere/embedding_intrinsic_isometric_local_dense}
				\caption{\label{fig:punctured_severed_sphere_intrinsic_isometric_embedding}}
			\end{subfigure}
		\end{centering}
		\begin{centering}
			\begin{subfigure}[b]{0.32\linewidth}
				\includegraphics[width=1\linewidth]{figures/Chapter_3/cross_punctured_2d_square_severed_sphere/standard_isomap_stress}
				\caption{\label{fig:punctured_severed_sphere_standard_isomap_stress}}
			\end{subfigure}
			\hfill
			\begin{subfigure}[b]{0.32\linewidth}
				\includegraphics[width=1\linewidth]{figures/Chapter_3/cross_punctured_2d_square_severed_sphere/stress_intrinsic_isomap_local_dense}
				\caption{\label{fig:punctured_severed_sphere_intrinsic_isomap_stress}}
			\end{subfigure}
			\hfill
			\begin{subfigure}[b]{0.32\linewidth}
				\includegraphics[width=1\linewidth]{figures/Chapter_3/cross_punctured_2d_square_severed_sphere/stress_intrinsic_isometric_local_dense}
				\caption{\label{fig:punctured_severed_sphere_intrinsic_isometric_stress}}
			\end{subfigure}
		\end{centering}
		\caption{\label{fig:punctured_severed_sphere} Punctured severed sphere (embedding). \protect\subref{fig:punctured_severed_sphere_intrinsic} Intrinsic space. 
		\protect\subref{fig:punctured_severed_sphere_observed} Observed space. 
		\protect\subref{fig:punctured_severed_sphere_intrinsic_metric} Intrinsic metric. 
		\protect\subref{fig:punctured_severed_sphere_intrinsic_dist_est} Intrinsic distance approximation.
		\protect\subref{fig:punctured_severed_sphere_intrinsic_dist_est_knn} Intrinsic distance approximation - $k$-NN only.
		\protect\subref{fig:punctured_severed_sphere_standard_isomap_embedding} Standard Isomap embedding.
		\protect\subref{fig:punctured_severed_sphere_intrinsic_isomap_embedding} Intrinsic Isomap embedding.
		\protect\subref{fig:punctured_severed_sphere_intrinsic_isometric_embedding} Intrinsic Isometric embedding.
		\protect\subref{fig:punctured_severed_sphere_standard_isomap_stress} Standard Isomap Stress.
		\protect\subref{fig:punctured_severed_sphere_intrinsic_isomap_stress} Intrinsic Isomap Stress.
		\protect\subref{fig:punctured_severed_sphere_intrinsic_isometric_stress} Intrinsic Isometric Stress.
		}
	\end{figure}

	As expected we see that the distance approximation is indeed valid for short distances for which the manifold is approximately linearly distorted, additionally we see that geodesic distance approximation does not suffer from this since geodesics can be calculated using only local distance information. Finally we see that the  non-convexity of the intrinsic dataset does indeed cause a distortion in the intrinsic isometric embedding when compared to the ground truth.

	\subsubsection{Metric estimation}
	
	Next we analyze the effect of using an intrinsic metric estimated from the observed data as opposed to using the exact intrinsic metric (as was the case in \cref{ssec:simulated_data_Intrinsic_isometric_embedding}). To do so we use the same example used in the previous sub-section under the setting described in \cref{ssec:Intrinsic-isotropic-GMM} with either $N_{i}=5$ or $N_{i}=200$ measurements made at each sample point, sampled from a isotropic Gaussian probability distribution centered at the sample point with intrinsic variance $\sigma_{int}^{2}=0.03^{2}$ in the intrinsic space. Additionally observation noise is added with variance $\sigma_{obs}^{2}=0.03^{2}$.
	
	In \cref{fig:metric_punctured_severed_sphere} we compare the trivial local estimation (as described in \cref{ssec:Intrinsic-isotropic-GMM}) and the global estimation approach implemented via a \ac{ANN} which is suggested in this paper (\cref{sec:Intrinsic-Metric-Estimation}). For each metric estimation method we plot, similarity to the previous sub-section, a visualization of the metric via ellipsis, a scatter plot of the true intrinsic distance against the approximated intrinsic distances, the resulting low-dimensional embedding and the scatter plot of the Euclidean distances in the embedding compared to the true intrinsic distance ,including a calculation of the Stress value. These are produced for the case of local estimation with dense sampling $N_{i}=200$ (\cref{fig:1a}, \cref{fig:1d}, \cref{fig:1g}, \cref{fig:1j}), for the case of sparse sampling (\cref{fig:1b}, \cref{fig:1e}, \cref{fig:1h}, \cref{fig:1k}) and finally for our sugestedd global metric estimation method (\cref{fig:1c}, \cref{fig:1f}, \cref{fig:1i}, \cref{fig:1l}).
	
	Similar results on several other synthetic data sets are provided as supplementary material \cref{ssec:Additional-results-manifold-learning}.
	
	\begin{figure}[h]
		\begin{centering}
			\begin{subfigure}[b]{0.32\linewidth}
					\includegraphics[width=1\textwidth]{figures/Chapter_4/cross_punctured_2d_square_severed_sphere/metric_local_dense}
					\caption{\label{fig:1a}}
			\end{subfigure}\hfill
			\begin{subfigure}[b]{0.32\linewidth}
					\includegraphics[width=1\textwidth]{figures/Chapter_4/cross_punctured_2d_square_severed_sphere/metric_local}
					\caption{\label{fig:1b}}
			\end{subfigure}\hfill
			\begin{subfigure}[b]{0.32\linewidth}
					\includegraphics[width=1\textwidth]{figures/Chapter_4/cross_punctured_2d_square_severed_sphere/metric_net}
					\caption{\label{fig:1c}}
			\end{subfigure}
		\end{centering}
		\begin{centering}
			\begin{subfigure}[b]{0.32\linewidth}
					\includegraphics[width=1\textwidth]{figures/Chapter_4/cross_punctured_2d_square_severed_sphere/dist_local_dense}
					\caption{\label{fig:1d}}
			\end{subfigure}\hfill
			\begin{subfigure}[b]{0.32\linewidth}
					\includegraphics[width=1\textwidth]{figures/Chapter_4/cross_punctured_2d_square_severed_sphere/dist_local}
					\caption{\label{fig:1e}}
			\end{subfigure}\hfill
			\begin{subfigure}[b]{0.32\linewidth}
					\includegraphics[width=1\textwidth]{figures/Chapter_4/cross_punctured_2d_square_severed_sphere/dist_net}
					\caption{\label{fig:1f}}
			\end{subfigure}
		\end{centering}
		\begin{centering}
			\begin{subfigure}[b]{0.32\linewidth}
					\includegraphics[width=1\textwidth]{figures/Chapter_4/cross_punctured_2d_square_severed_sphere/embedding_intrinsic_isometric_local_dense}
					\caption{\label{fig:1g}}
			\end{subfigure}\hfill
			\begin{subfigure}[b]{0.32\linewidth}
					\includegraphics[width=1\textwidth]{figures/Chapter_4/cross_punctured_2d_square_severed_sphere/embedding_intrinsic_isometric_local}
					\caption{\label{fig:1h}}
			\end{subfigure}\hfill
			\begin{subfigure}[b]{0.32\linewidth}
					\includegraphics[width=1\textwidth]{figures/Chapter_4/cross_punctured_2d_square_severed_sphere/embedding_intrinsic_isometric_net}
					\caption{\label{fig:1i}}
			\end{subfigure}
		\end{centering}
		\begin{centering}
			\begin{subfigure}[b]{0.32\linewidth}
					\includegraphics[width=1\textwidth]{figures/Chapter_4/cross_punctured_2d_square_severed_sphere/stress_intrinsic_isometric_local_dense}
					\caption{\label{fig:1j}}
			\end{subfigure}\hfill
			\begin{subfigure}[b]{0.32\linewidth}
					\includegraphics[width=1\textwidth]{figures/Chapter_4/cross_punctured_2d_square_severed_sphere/stress_intrinsic_isometric_local}
					\caption{\label{fig:1k}}
			\end{subfigure}\hfill
			\begin{subfigure}[b]{0.32\linewidth}
					\includegraphics[width=1\textwidth]{figures/Chapter_4/cross_punctured_2d_square_severed_sphere/stress_intrinsic_isometric_net}
					\caption{\label{fig:1l}}
			\end{subfigure}
		\end{centering}
	
	\caption{Punctured severed sphere (metric estimation).
		\protect\subref{fig:1a} True intrinsic metric.
		\protect\subref{fig:1b} Locally estimated intrinsic metric.
		\protect\subref{fig:1c} Net learned intrinsic metric.
		\protect\subref{fig:1d} Distance estimation using true metric.
		\protect\subref{fig:1e} Distance estimation using locally estimated metric.
		\protect\subref{fig:1f} Distance estimation using net estimated metric.
		\protect\subref{fig:1g} Embedding using true intrinsic metric.
		\protect\subref{fig:1h} Embedding using locally learned intrinsic metric.
		\protect\subref{fig:1i} Embedding using net learned intrinsic metric.
		\protect\subref{fig:1j} Embedding stress using true intrinsic metric.
		\protect\subref{fig:1k} Embedding stress using locally estimated intrinsic metric.
		\protect\subref{fig:1l} Embedding stress using net learned intrinsic metric}
	
	
	\label{fig:metric_punctured_severed_sphere}
	\end{figure}

	We see that the estimations in the sparse case are ``noisy'' and can sometimes change abruptly between similar locations on the manifold. This noisiness in the estimated distances adversely effects the intrinsic distance estimation is the cornerstone of the algorithm suggested in \cref{sec:Intrinsic-isometric-manifold-learning} this of course damages the results of the intrinsic-isometric learned representation. Similar results on several other synthetic data sets are provided as supplementary material.
	
	\subsection{Localization in sensor networks}
	\label{ssec:localization}
	
	In \cref{sec:motivation} we provided initial motivation for our work through the simple and intuitive example of localization in sensor networks. We now revisit this example and discuss in detail how the manifold learning algorithm proposed in this work can be applied to this problem. 
	
	In this experiment, we simulate positioning of an agent using observations with an unknown model, which is intended to represent the setting encountered for indoor positioning. Through this experiment we examine the advantages of intrinsic geometry preservation and demonstrate its relevance to complex, high-dimensional and realistic scenarios.
	
	\subsubsection{Experiment setting}
	\label{sssec:Experiment-setting}
	
	The experimental setting is as follows: An agent is allowed to be located within a compact, path-connected subset $\mathcal{X}$ of $\mathbb{R}^{2}$ which represents a closed indoor environment. The shape of $\mathcal{X}$ used in this experiment is depicted in \cref{fig:Indoor-environment-shape}. At each point $\mathbf{x}\in\text{\ensuremath{\mathcal{X}}}$ a number of measurements of different modalities (which will be described later) are observed. These observations are such that they are only effected by the locations where the measurements are made (and possibly some additional noise which is assumed to be uncorrelated with the location). Such observations are made in enough different locations so that $\mathcal{X}$ is completely covered. The dimension of the intrinsic vector space is $n=2$ corresponding to the two dimensional physical space and the dimension of the observation space depends on the dimensionality of observation function output but is in general high-dimensional. 
	
	\begin{figure}[h]
		\begin{centering}
			\includegraphics[scale=0.5]{figures/Chapter_5/apt_geometry}
		\end{centering}
		\caption{Outline of the intrinsic manifold which represents a confined indoor environment \label{fig:Indoor-environment-shape}}
	\end{figure}
	
	As discussed in \cref{sec:Intrinsic-Metric-Estimation}, in order to uncover the intrinsic metric of the manifold from the observed data, we require the intrinsic data sampling to adhere to some known structure. For this experiment the intrinsic sampling is assumed to be acquired by the use of a rigid sensor array, as described in \cref{ssec:Rigid-sensor-array}. The sensor array used consists of measurement points with a structure that corresponds to measurements in two orthogonal directions with $15cm$ distance between measurements as illustrated in \cref{fig:sensor_array_localization} and depicts the points at which measurements were performed for a sub-set of 13 sampling points.
	

	\subsubsection{Sensor modalities}
	
	To stress the fact that our algorithm is invariant with respect to the sensor modality (i.e to the specific observation function used), we perform simultaneous observations using multiple
	different modalities. 
	
	In what follows we only discuss two vision related modalities, however results for other modalities are included in the supplementary material \cref{ssec:additional-results-localization}. 
	
	
	\subsubsection*{Color camera}
		\label{sssec:Color-camera}
	
	This modality simulates a camera mounted on the agent which shows the environment from the point of view of the agent. To accomplish this, we used ``Blender'', a professional, freely available and open-source 3-dimensional graphics software. Using ``Blender'' \url{www.blender.org}, we constructed a 3-dimensional model mimicking the interior of an apartment. The created model is shown in \cref{fig:3D-model-in}. The regions of the model in which there are no objects and in which the simulated agent is allowed to move, correspond to the shape of $\mathcal{X}$ presented in \cref{fig:Indoor-environment-shape}.

	\begin{figure}[h]
		\begin{centering}
			\begin{subfigure}[b]{0.45\columnwidth}%
				\includegraphics[width=1\textwidth]{figures/Chapter_5/blue_print}
			\end{subfigure} \hfill
			\begin{subfigure}[b]{0.45\columnwidth}%
				\includegraphics[width=1\textwidth]{figures/Chapter_5/blue_print_3d}
			\end{subfigure}
		\end{centering}
		\caption{3-dimensional model in Blender\label{fig:3D-model-in}}
	\end{figure}
	
	``Blender'' allows us to render an image of the model as seen via a virtual camera. Using this ability, we produce a set of 360 degree panoramic color images of size $128\times256$ pixels, taken from the point of view of the agent as seen in \cref{fig:Room-from-Agent's}. These images serve as an observation of the location of the agent. Slight variations in the location of the agent cause slight variations in the point of view of the camera and therefore in the produced image, as seen in \cref{fig:sensor_array_localization}. These slight observed variations, combined with our assumption about the intrinsic structure of the data allow us to infer the local intrinsic metric.
	
	\begin{figure}[h]
		\begin{centering}
			\begin{subfigure}[t]{0.47\columnwidth}
				\includegraphics[width=1\textwidth]{figures/Chapter_5/roomba_view_1}
			\end{subfigure}\hfill
			\begin{subfigure}[t]{0.47\columnwidth}
				\includegraphics[width=1\textwidth]{figures/Chapter_5/roomba_view_2}
			\end{subfigure}
		\end{centering}
	
		\begin{centering}
			\begin{subfigure}[t]{0.47\columnwidth}
				\includegraphics[width=1\textwidth]{figures/Chapter_5/roomba_view_3}
			\end{subfigure}\hfill
			\begin{subfigure}[t]{0.47\columnwidth}
				\includegraphics[width=1\textwidth]{figures/Chapter_5/roomba_view_4}
			\end{subfigure}
		\end{centering}
		\caption{Samples of generated panoramic color images}
		\label{fig:Room-from-Agent's}
	\end{figure}
	
	\begin{figure}[h]
		\begin{centering}
			\includegraphics[width=1\textwidth]{figures/Chapter_5/sensor_array/small_var}
		\end{centering}
		\caption{Observation points using a rigid sensor array for a subset of 13 sample points. On the right, one can see the effect of slight variations in agents position on the viewed panoramic images \label{fig:sensor_array_localization}}
	\end{figure}
	
	One of the requirements of the algorithm presented in \cref{sec:Intrinsic-isometric-manifold-learning}, was that the observations need to only be a function of the latent variable, which in this case is the 2-dimensional location of the agent. Panoramic images produced at the same location but starting from different angles would be cyclically rotated with respect to each other, thus violating this requirement. To overcome this, we make these observations invariant to cyclical rotation on the horizontal axis. We implement this in the frequency domain by applying a Fourier transform to each frame and then estimating and removing the linear phase in the horizontal direction. Since cyclical rotations in the horizontal axis are equivalent to an addition of linear phase in the Fourier domain, this makes the observation invariant to the initial cyclical rotation. 
	
	In order to reduce the initial dimensionality of the data, \ac{PCA} was performed and only the first 100 principle components were taken since these practically contained all the energy/power in the data.
	
	Images are used as inputs to our algorithm since this represents a possible realistic setting and since it is a non-liner sensor modality which is easy to simulate using 3-dimensional
	modeling software. We wish to emphasis however, that after the per-processing stage in which these images are made invariant to cyclical rotations, the input is no longer treated as an image, and our proposed algorithm uses no additional image or computer vision related computation on the input. This invariance of the algorithm to the input type, allows us to perform the additional stage of dimensionality reduction using \ac{PCA} which would otherwise not be possible since it ``strips'' the input of its image structure.
	
	\subsubsection*{Depth Camera}
	
	An additional sensor modality, which is also produced using ``Blender'', is a gray-scale depth map, where the gray level at each pixel represents the distance, in that pixels direction, from the observing camera to the nearest object. Several examples of such images are shown in \cref{fig:Panoramic-depth-images}. Since the geometry of the image acquisition model is similar, the need for imposing invariance to cyclical rotation in the horizontal axis arises again and we perform the same per-processing stages described in \cref{sssec:Color-camera}. The dimensionality of the input is also reduced using \ac{PCA} to 40 principle component since this accounts for almost all of the observed energy/power in the data.

	\begin{figure}[h]
		\begin{centering}
			\begin{subfigure}[t]{0.47\columnwidth}
				\includegraphics[width=1\textwidth]{figures/Chapter_5/roomba_depth_1}
			\end{subfigure}\hfill
			\begin{subfigure}[t]{0.47\columnwidth}
				\includegraphics[width=1\textwidth]{figures/Chapter_5/roomba_depth_2}
			\end{subfigure}
		\end{centering}
	
		\begin{centering}
			\begin{subfigure}[t]{0.47\columnwidth}
				\includegraphics[width=1\textwidth]{figures/Chapter_5/roomba_depth_3}
			\end{subfigure}\hfill
			\begin{subfigure}[t]{0.47\columnwidth}
				\includegraphics[width=1\textwidth]{figures/Chapter_5/roomba_depth_5}
			\end{subfigure}
		\end{centering}
		\caption{Samples of generated panoramic depth maps \label{fig:Panoramic-depth-images}}
	\end{figure}
	
		
	\subsubsection{Results}
	
	For each of the four modalities described above $N=1000$ intrinsic points were sampled, 3 measurement were made around each such location using the sensor array described, a 2-dimensional embedding was constructed using our proposed algorithm and using the standard Isomap methods. Results are presented in \cref{fig:Color-image-observations}, \cref{fig:Depth-image-observations}.
			
	\begin{figure}[h]
		\begin{centering}
			\begin{subfigure}[t]{0.47\columnwidth}%
				\includegraphics[width=1\textwidth]{figures/Chapter_5/color/intrinsic}
				\caption{\label{fig:3a}}
			\end{subfigure}\hfill
			\begin{subfigure}[t]{0.47\columnwidth}%
				\includegraphics[width=1\textwidth]{figures/Chapter_5/color/regular_isomap_embedding}
				\caption{\label{fig:3b}}
			\end{subfigure}
		\end{centering}
		\begin{centering}
			\begin{subfigure}[t]{0.47\columnwidth}%
				\includegraphics[width=1\textwidth]{figures/Chapter_5/color/dist_local}
				\caption{\label{fig:3c}}
			\end{subfigure}\hfill
			\begin{subfigure}[t]{0.47\columnwidth}%
				\includegraphics[width=1\textwidth]{figures/Chapter_5/color/dist_local_knn}
				\caption{\label{fig:3d}}
			\end{subfigure}
		\end{centering}
		\begin{centering}
			\begin{subfigure}[t]{0.47\columnwidth}%
				\includegraphics[width=1\textwidth]{figures/Chapter_5/color/intrinsic_isometric_local}
				\caption{\label{fig:3e} \label{fig:bend1}}
			\end{subfigure}\hfill
			\begin{subfigure}[t]{0.47\columnwidth}%
				\includegraphics[width=1\textwidth]{figures/Chapter_5/color/intrinsic_isometric_local_stress}
				\caption{\label{fig:3f}} 
			\end{subfigure}
		\end{centering}
		\caption{Color image observations\label{fig:Color-image-observations}. 
			\protect\subref{fig:3a} Intrinsic space. 
			\protect\subref{fig:3b} Embedding using standard Isomap. 
			\protect\subref{fig:3c} Intrinsic Euclidean distance estimation. 
			\protect\subref{fig:3d} Intrinsic Euclidean distance estimation ($k$-NN). 
			\protect\subref{fig:3e} Intrinsic-isometric embedding. 
			\protect\subref{fig:3f} Euclidean distance discrepancy and stress in resulting embedding}
	\end{figure}
	
	\begin{figure}[h]
		\begin{centering}
			\begin{subfigure}[t]{0.47\columnwidth}
				\includegraphics[width=1\textwidth]{figures/Chapter_5/depth/intrinsic}
				\caption{\label{fig:4a}}
			\end{subfigure}\hfill
			\begin{subfigure}[t]{0.47\columnwidth}
				\includegraphics[width=1\textwidth]{figures/Chapter_5/depth/regular_isomap_embedding}
				\caption{\label{fig:4b}}
			\end{subfigure}
		\end{centering}
		\begin{centering}
			\begin{subfigure}[t]{0.47\columnwidth}
				\includegraphics[width=1\textwidth]{figures/Chapter_5/depth/dist_local}
				\captionsetup{justification=centering}
				\caption{\label{fig:4c}}
			\end{subfigure}\hfill
			\begin{subfigure}[t]{0.47\columnwidth}
				\includegraphics[width=1\textwidth]{figures/Chapter_5/depth/dist_local_knn}
				\caption{\label{fig:4d}}
			\end{subfigure}
		\end{centering}
		\begin{centering}
			\begin{subfigure}[t]{0.47\columnwidth}%
				\includegraphics[width=1\textwidth]{figures/Chapter_5/depth/intrinsic_isometric_local}
				\caption{\label{fig:4e} \label{fig:bend2}}
			\end{subfigure}\hfill
			\begin{subfigure}[t]{0.47\columnwidth}
				\includegraphics[width=1\textwidth]{figures/Chapter_5/depth/intrinsic_isometric_local_stress}
				\caption{\label{fig:4f}}
			\end{subfigure}
		\end{centering}
		\caption{Depth image observations \label{fig:Depth-image-observations}. 
			\protect\subref{fig:4a} Intrinsic space. 
			\protect\subref{fig:4b} Embedding using standard Isomap. 
			\protect\subref{fig:4c} Intrinsic Euclidean distance estimation. 
			\protect\subref{fig:4d} Intrinsic Euclidean distance estimation ($k$-NN). 
			\protect\subref{fig:4e} Intrinsic-isometric embedding. 
			\protect\subref{fig:4f} Euclidean distance discrepancy and stress in resulting embedding}
	\end{figure}

	Our proposed algorithm accurately retrieves the intrinsic structure for all observation modalities. This is evident by fact that the intrinsic-isometric embedding structure is almost identical to the structure of the sampled points in the intrinsic space (can be observed for both modalities by comparing \cref{fig:3a} to \cref{fig:3e} and by comparing \cref{fig:4a} to \cref{fig:4e}) and from the fact that most inter-point distances in the final embedding closely approximate the true intrinsic Euclidean distance which also leads to a low stress value for the embedding (as can be observed for both modalities in \cref{fig:3f} and \cref{fig:4f}). The success of the embedding stems form the fact that the estimated intrinsic metric allows for a good estimation of short-range intrinsic distances (as can be observed for both the modalities in \cref{fig:3c}, \cref{fig:3d}, \cref{fig:3c} and \cref{fig:3d}). We notice that since the observation functions used are not locally isometric, standard Isomap fails to retrieve the intrinsic structure of the data or to even provide a 2-dimensional parameterization of the intrinsic space (as can be observed for all the modalities in \cref{fig:3b} \cref{fig:4c}). Isomap does, for the most part, preserve proximity on a local scale (can be observed by points with similar colors being embedded close to each other) but global structure is not preserved (as can be observed for all the modalities by comparing \cref{fig:3a} and \cref{fig:3b} and \cref{fig:4a} and \cref{fig:4b}). Since standard Isomap is non-intrinsic, it is effected by the modality of the observation and we receive a different embeddings for different sensor modalities (as can be observed by comparing \cref{fig:3b} and \cref{fig:3d}).
	
	One noticeable weak point of our algorithm, which manifests slightly in these examples, occurs around the coordinate $\left(9,4\right)$ in the true intrinsic space (\cref{fig:3a} and \cref{fig:4a}). This region corresponds to a narrow area in the apartment model in which there are not a lot of sample points. Errors in distance estimations for these points are not ``balanced'' or ``countered'' by distance constrains in other regions of the apartment since no other distance constraints influence this region. This leads to a slight distortion in the embedding of this region which causes a ``bend'' in the global structure of the embedding (as seen in \cref{fig:bend1} and \cref{fig:bend2}).
	
	These results show that our proposed intrinsic-isometric dimensionality reduction algorithm could be successfully applied to the problem of localization and mapping. While this application might fall into the much researched subject of indoor mapping and localization, we wish to remark that we make no claim that this algorithm is superior or even comparable to existing algorithms tailored to specific measurement modalities or to machine learning tools trained on labeled data-sets. Our algorithms advantage that it is completely unsupervised and modality invariant which makes it especially suitable for setting where one wants to use an automatic algorithm for localization or/and when the observation model is unknown; which is often the case with indoor localization.

	
	\externaldocument{paper.tex}
	\section{Conclusions}
		\label{sec:conclusions}
		
		In this work, we addressed the subject of intrinsic and isometric
	manifold learning. We first showed that the need for methods, which
	preserve the latent Euclidean geometric structure of data manifolds,
	arises naturally when inherently low-dimensional systems are observed
	via high-dimensional non-linear measurements or observations. We presented
	a new manifold learning algorithm which uses local properties of the
	observation function to estimate a local intrinsic metric (the ``push-forward''
	metric of the observation function). This metric was then used to
	estimate intrinsic geometric proprieties of the data directly from
	the observed manifold as if they were calculated in the low-dimensional
	latent space. We discussed a few settings under which estimation of
	the required local observation function properties is possible from
	the observed data itself. Unfortunately, we recognized that due to
	their local nature, these metric estimation methods are not sufficiently
	robust to high curvature of the observation function as well as to
	noise. We suggested to overcome this by parameterizing all estimated
	intrinsic metrics on the observed manifold as the output of a single
	\ac{ANN} and performing a signal optimization in order to estimate
	all local intrinsic metric simultaneously. This procedure has probabilistic
	reasoning and was shown to be equivalent to maximum-likelihood estimation
	under a certain statistical model. We showed that this couples the
	metric estimations at different points on the manifold, regularizes
	the estimation and imposes smoothness of the estimated metric. We
	discussed the possibility of additionally imposing regularization
	on the estimation by the net structure and weight decay terms, which
	is common practice in \acp{ANN}. By combining a robust intrinsic
	metric estimation method and an algorithm which can use these metrics
	to build an intrinsic and isometric embedding, we devised an algorithm,
	which can automatically recover the geometric structure of a latent
	low-dimensional manifold from its observations via an unknown non-linear
	high-dimensional function. Finally we focused on the example of mapping
	and positioning an agent using a sensor network of unknown nature
	and modalities. We showed that our proposed algorithm can recover
	the structure of the space in which the agent moves and can correctly
	position the agent within that space. Due to the intrinsic nature
	of our method, it can perform this mapping and positioning without
	the need for prior knowledge of a measurement model to explain the
	connection between the observed measurements and the position of the
	agent. This invariance to the type of measurement used, was shown
	to be suitable in a setting such as indoor positioning where the exact
	measurement model is usually unknown.
	
	It is evident that our method outperforms the local metric estimation
	approach for all the tested examples and enables us to learn intrinsically-isometric
	representation for broader cases problems given very sparse sampling
	and observation noise of the same scale of the intrinsic variance
	which is used to estimate the intrinsic metric. 
	
	The use of neural networks as regression functions. provides a powerful
	regularizing factor in intrinsic metric estimation. Their general
	structure with the plethora of optimization algorithms, regularization
	tweaks and efficient implementation methods make them a very good
	candidate for a parametric function family to be used in the problem
	of metric estimation.
	
	The estimation method described in this chapter was proven to have
	probabilistic reasoning when assuming an intrinsic-isometric \ac{GMM},
	however it also has an intuitive interpretation in a more general
	case. As can be observed in \cref{eq:KL}, the network tries approximate
	locally calculated estimations of the intrinsic metric while adhering
	to a globally smooth and regularized model. This can serve as a good
	general heuristic method to impose global smoothness and regularization
	on locally estimated intrinsic metrics under other probabilistic or
	non-probabilistic setting such as the one presented in
	\cref{ssec:Intrinsic-isotropic-GMM}.
	
	A somewhat similar method was presented in \cite{bengio2004non} where
	an \ac{ANN} was also used in order to parametrize the spanning vectors
	of the local tangent plane to a manifold. The only criterion provided
	for the estimation was minimization of the distance of observed points
	from the local tangent plane. The estimation was not given probabilistic
	reasoning and the estimated spanning vector had no physical meaning
	and did not provide any metric interpretation of information. In our
	work, we not only retrieve the tangent plane to the manifold at each
	point but also an intrinsic metric and we do so with probabilistic
	reasoning.
	
	In the case of neural networks ``the devil is in the details'' and
	implementation plays a large role in the final performance of the
	trained network and the subjects of neural network training and regularization
	are vast topics which are out of the scope of this work. Our main
	contribution is to show that neural networks can, in principle be
	used, to regularize estimations and in this specific case intrinsic
	metric estimation and we do not attempt to claim that this architecture
	and optimization procedure achieve an optimal solution.
	
	The global estimation approach presented in this chapter has some
	secondary advantages in addition to providing additional robustness
	to the intrinsic metric estimation when compared to local metric estimation
	methods. As opposed to local methods which only estimate the intrinsic
	metric for observed sample points, the regression approach produces
	an estimation of the intrinsic metric on the whole observed space.
	This can be possibly used to generate additional points in between
	existing points, thus artificially increasing the sample density.
	This can improve both the short-range intrinsic distance estimation
	described in \cref{ssec:Intrinsic-geometry-approximation} and
	the geodesic distance estimation described in \cref{ssec:Global-geometry-approximation}.
	Both effects should improve the results of the algorithm presented
	in \cref{sec:Intrinsic-isometric-manifold-learning}.
	
	In \cite{bengio2004non} it was shown that by leaning the tangent
	space to the manifold at each point, one can ``walk'' on the manifold
	by making infinitesimal steps each time in the tangent space. Since
	with the method suggested in this chapter, we do not only estimate
	the tangent space but the intrinsic metric as well, we know how far
	we have gone in terms of the distance, an ability that might be relevant
	to applications such as non-linear interpolation \cite{bregler1995nonlinear}.
	
	
	\subsection{Pre-processing}
		\label{subsec:Pre-processing}
	
	Since our approach does not assume anything about the structure of
	the observed data, one can perform pre-processing stages for reducing
	the dimensionality of the data and possibly removing noise without
	worrying about maintaining the structure of the data. This has the
	advantage of enabling the use of other existing dimensionality reduction
	methods as pre-processing stages to our algorithm. For example, when
	localization and mapping were performed using image data, the algorithm
	did not treat the data as images, which allowed us to use \ac{PCA}
	to lower the dimensionality of the image data greatly. This would
	not be possible for vision based algorithms which exploit the image
	structure since the application of \ac{PCA} would remove this structure
	and such algorithm could no longer be applied. For our method the
	image structure is unimportant and one can do without it.
	
	\subsection{Practical application to indoor-positioning}
		\label{subsec:Practical-application-to}
	
	The ability to localize sets of measurements without knowledge of
	the model connection positions and measurements is a problem encountered
	in real life indoor positioning due to the complexity of the models
	and the variations, as described in \cref{sec:introduction}.
	Learning a model for each specific setting requires large amount of
	labeled data. Our algorithm might replace the need for labeled data
	acquisition with a simple prior assumption on the unlabeled measurement
	acquisition process. 
	
	This has been showed to work in theory however it has yet to be applied
	to a realistic indoor positioning system. Basic physical experiments
	using a randomly walking agent in 2-dimensional space are underway.
	In these experiments a robotic iRobot Roomba programmable robotic
	vacuum cleaner was controlled by a Raspberry-Pi mini computer and
	performed a random walk inside a 2-dimensional region. Signal acquisition
	was performed either my measuring the \ac{RSS} from WiFi stations
	or via a wide angle lens which produced 360 degree images. Results
	from the experiments will be published when concluded. Images from
	these experiments are shown in \cref{fig:Roomba-experiments}.
	
	\iffalse
	\begin{figure}[h]
		\begin{centering}
			\begin{minipage}[t]{0.45\columnwidth}%
				\begin{flushleft}
					\subfloat[\ac{RSS} measurement]{\begin{centering}
							\includegraphics[width=1\textwidth]{figures/Chapter_6/Roomba_wifi}
							\par\end{centering}
					}
					\par\end{flushleft}%
			\end{minipage}\hfill{}%
			\begin{minipage}[t]{0.45\columnwidth}%
				\subfloat[Panoramic camera]{\begin{centering}
						\includegraphics[width=1\textwidth]{figures/Chapter_6/Roomba_pano}
						\par\end{centering}
				}%
			\end{minipage}
			\par\end{centering}
		\caption{Roomba experiments \label{fig:Roomba-experiments}}
	\end{figure}
	\fi
	
	
	\subsection{Generalized localization problems}
	
	Throughout this work and in \cref{sec:introduction} and
	\cref{ssec:localization} specifically, we discussed the problem
	of localization in sensor networks and presented it as an example
	of a problem which requires an intrinsic and isometric manifold learning
	method in order to fully utilize the observed data and recover a mapping
	of the physical space and a localization of an agent within this space.
	However, this should not be regarded as a specific problem (although
	it is by itself an important one) but as a general prototypical problem,
	for which physical localization in 2-dimensional or 3-dimensional,
	is just one intuitive manifestation. Indeed, any problem where the
	uncovering of the global geometric structure of a low-dimensional
	parameter space is desirable, can be regarded as a ``localization
	`` problem. Trivially, this can occur when structure recovery is
	by itself a goal and one can think of many examples where the global
	structure of the parameter space is important or has some special
	meaning (for example for imaging or molecule structure recovery),
	yet this global structure is also crucial when one requires the intrinsic
	structure of a low-dimensional vector space. The process of observation
	via a non-linear function causes the data to lose its global structure,
	and the initially ``flat'' manifold becomes curved. This hinders
	our ability to perform operations on the parameter space which require
	a vector space structure such as addition, subtraction, division and
	multiplication. This are the basic operation required for more high-level
	processing such as averaging, clustering, interpolation, learning
	using $k$-NN (especially when samples are sparse and neighbors are
	far from each other) etc. Embedding the data back into a low-dimensional
	space recovers the vector space structure of the original latent space
	and enables these operations.
	
	\subsection{Automatic labeled data acquisition}
	
	Supervised machine learning uses labeled data (consisting of pairs
	of input objects and desired output values) and attempts to ``learn''
	or infer a functional connection between the two. Learning using more
	labeled data increases the learners ability to unseen examples and
	usually leads to better performance of the trained model. Unfortunately
	acquisition of labeled data is usually non-trivial, expensive and
	requires an already existing method to correctly label data. Our proposed
	approach can be seen as a method for automatic acquisition of labels
	as it operates in a completely unsupervised manner and produces data
	labeling as an output. The ability to produce labeled data can thus
	be reduced to the task of exploring the parameter space of the system
	under some restriction or statistical model which allows us to calculate
	a local intrinsic metric as described in this work. 
	
	This automatic acquisition of labeled data can be much easier in many
	cases than producing labeled data using an existing labeling method.
	As an example we return to the localization example discussed in depth
	in \cref{ssec:localization}. To acquire labeled data in
	the described setting, one would be required to perform a large set
	of observations (color images, depth maps, \ac{RSS} values etc.)
	and provide for each such observation the 2-dimensional coordinates
	at which they were taken. This procedure requires time, patience and
	an existing way to measure the correct coordinates. This becomes even
	more complicated if one considers localization in 3-dimensional space
	where accurate positioning requires special equipment. Alternatively,
	with our approach, one could use a randomly waking agent or a simple
	sensor array (as described in \cref{sec:Intrinsic-Metric-Estimation})
	to acquire a large set of unlabeled data, and then use our proposed
	algorithm to retrieve an approximation of their labels. One can then
	use a supervised machine learning algorithm in order to infer the
	label of new observations. In the application described in \cref{ssec:localization}, once labeling of a large set of images
	from the apartment space is acquired using our algorithm, one can
	feed these to a vision based algorithm which can be robust to variation
	in appearance of the space, such as changes of illumination, obstructions,
	or even some changes in the structure of the apartment due to movement
	of some objects.
	
	\subsection{Holistic approach}
	
	The algorithm presented in this work suggests a two-stage solution
	for intrinsic-isometric embedding of manifolds in low-dimensional
	Euclidean space:
	\begin{enumerate}
		\item Recovering local intrinsic metrics using prior-knowledge and assumptions
		about the intrinsic process and using them to calculate intrinsic
		inter-point distances.
		\item Constructing and embedding of the observed points into a low-dimensional
		space, where the Euclidean distance respects the calculated intrinsic
		inter-point distances.
	\end{enumerate}
	If the first stage of this approach does not result in accurate enough
	estimations of the intrinsic geometry, the second stage can potentially
	fail. Combining the two stages described above into a one stage embedding
	method could be beneficial. A possible way to implement this is to
	directly learn an embedding function of the observed data into a low-dimensional
	space such that the embedding best fits prior-knowledge or assumption
	we have about the intrinsic system. This could be implemented for
	example with an \ac{ANN} for which the cost function will be the
	likelihood that the embedding originated from the statistical model
	of the intrinsic latent system. Such an implementation would however,
	rarely converge to minima which represent a good embedding of the
	data without being given an initial solution which is close enough
	to the true structure. One can however use the embedding received
	via the presented two stage approach as an initial target embedding
	for such an \ac{ANN}, and then further maximize the likelihood of
	the embedding directly on the embedding space. 
	
	\subsection{Sensor fusion}
	
	The fact that our proposed algorithm is invariant to the observation
	function, allows sensor fusion. Measurements from different sensors
	modalities can be simply concatenated, creating a higher-dimensional
	observation function. Although this should work in theory, it is clear
	that if the sensors are improperly scaled with respect to each other
	this will not be optimal. Further research should be invested in considering
	the proper way to fuse different measurement modalities in a way that
	optimizes the results of the algorithm. 
	
	\subsection{Towards data-driven Taken's embedding}
	
	Our algorithm attempts to produce an embedding of the observed manifold
	into a low-dimensional space, where similar observations are embedded
	to similar locations in the constructed embedding space. This causes
	a problem if different intrinsic points have identical or very similar
	observation values. In this work we resolved this by assuming that
	the observation function is invertible. One practical approach to
	deal with this situation is to add observations until any such similar
	points are distinguished. However, when a set of observations is given
	and one cannot add additional ``physical'' observations we require
	a different way required. If temporal information about the dynamical
	observation process is available one can, in principle, distinguish
	points by their respective past and future observations. Such a separation
	between points would be especially beneficial in the presence of high
	levels of observation noise which might make originally distinct measurement
	values similar to each other. Our proposed algorithm currently uses
	temporal side-information only for the stage of estimating the intrinsic
	metric and does not incorporate this information for the purpose of
	the final embedding into low-dimensional space. One possible approach
	for incorporating this information naturally into our algorithm is
	by the observation of several lagged measurements simultaneously,
	by looking far ``enough'' into the past, such that any two points
	become distinguishable. This was proven for the case of deterministic
	latent dynamical system by Taken's embedding theorem \cite{takens1981detecting}.
	
	\subsection{Implementation using vector-diffusion maps}
	
	Our proposed manifold learning algorithm computes global intrinsic
	geometric properties by using the shortest-path algorithm in order
	to approximate inter-point geodesic distances. The shortest-path algorithm
	operated by propagating the distance function from a source point
	across the manifold. A similar process of propagating local vector
	information (instead of scalar distance information) could be used
	to allow for direct long range intrinsic Euclidean distance estimation.
	A framework for vector diffusion on Riemannian manifold has been presented
	in \cite{singer2012vector} and could possibly be used in order to
	implement this. Euclidean distance estimation by vector diffusion
	should be unaffected by non-convexity of the intrinsic manifold and
	can allow for use of global distances making the final embedding more
	robust.
	
	\subsection{Incorporating labeled data}
		\label{subsec:Incorporating-labeled-data}
	
	The algorithm proposed in this work operates in a completely unsupervised
	manner. The only prior-knowledge used is the assumption about the
	intrinsic data structure, which allows for estimation of the local
	intrinsic metric. While this is a desired property of the algorithm,
	since labeled data is usually harder to gather relative to unlabeled
	data, it does not harness the advantages of labeled data if such data
	does exist. We mention two possible ways to incorporate existing labeled
	data. Labeled data can be used as part of the intrinsic metric learning,
	where similar settings have been used in existing works in order to
	generate metrics which best describe existing label variations on
	data manifold \cite{yang2006distance,xing2003distance}. Alternatively,
	labeled data can be used in the embedding procedure itself, where
	one can easily add additional distance restriction known from labeled
	data. In the localization example given in \cref{ssec:localization}
	for example, we notice a slight distortion in the embedding on a global,
	long-range scale. This is because we only incorporate local distance
	information in the final embedding optimization, which does not fully
	restrict the global conformation in practical noisy situations. The
	short-distance estimations have errors, which induce slight distortions
	in the embedding. These distortions are not significant on small scales
	but they accumulate on larger scales and longer distances. Adding
	a few robustly measured long distances can be extremely beneficial
	for correcting this and even a very low number of such constraints
	could stabilize the embedding considerably, removing this accumulated
	error. Incorporating such robustly measured distances in the algorithm
	suggested in \cref{sec:Intrinsic-isometric-manifold-learning}
	can be done easily by giving such distance constraints much larger
	weights in the stress function presented in (\cref{eq:w-intrinisc_stress})
	than those of distances estimated from the observed data, which are
	less reliable.
	\appendix
	\externaldocument{paper.tex}


\section{Local metric estimation}
\label{sec:Local-intrinsic-metric}

We will now present a few settings in which one can approximate the intrinsic metric merely from observed data. All these settings require some assumptions about the structure of the data in the latent intrinsic space which generates the observed data-set. The need for such prior knowledge is not very surprising, since in order to ``look beyond'' the observed manifold and calculate properties of the intrinsic geometry in the latent space, we must somehow estimate what distortion was applied to the manifold by the observation function. A prior assumption about the structure of the latent data serves as a sort of gauge of this distortion and by comparing the observed data structure to the
assumed latent structure we can, under certain setting, infer the effect of the observation function.

In this section we focus on local methods of metric estimation, these methods operate on small patches of the manifold for which the observation function can be approximated as a linear function using its local Jacobian. The settings discussed below are not in any way unique and one can think of a multitude of scenarios in which priors assumptions about the latent system allow us to estimate the local distortion imposed on data by the unknown observation function.

\subsection{Rigid sensor array}
\label{ssec:Rigid-sensor-array}

Consider the values of an observation function $f:\mathbb{R}^{n}\to\mathbb{R}^{m}$ observed at $\mathbf{x}+\hat{\mathbf{u}}$. According to the Taylor series approximation:
\[
f\left(\mathbf{x}+\hat{\mathbf{u}}\right)=f\left(\mathbf{x}\right)+\frac{df}{dx}\left(\mathbf{x}\right)\hat{\mathbf{u}}+\mathcal{O}\left(\left\Vert \hat{\mathbf{u}}\right\Vert \right)
\]
If the difference between the two measurements $\left\Vert \hat{\mathbf{u}}\right\Vert $ is small with respect to the higher derivatives of the function $f$ we get that:
\[
f\left(\mathbf{x}+\hat{\mathbf{u}}\right)-f\left(\mathbf{x}\right)\approx\frac{df}{dx}\left(\mathbf{x}\right)\hat{\mathbf{u}}
\]
If one observes the function at $k$+1 intrinsic points $\mathbf{x},\mathbf{x}+\hat{\mathbf{u}}_{1},\ldots,\mathbf{x}+\hat{\mathbf{u}}_{k}$
we get a set of such equations :

\begin{align*}
	f\left(\mathbf{x}+\hat{\mathbf{u}}_{1}\right)-f\left(\mathbf{x}\right)= & \frac{df}{dx}\left(\mathbf{x}\right)\hat{\mathbf{u}}_{1}\\
	\vdots\\
	f\left(\mathbf{x}+\hat{\mathbf{u}}_{\text{k}}\right)-f\left(\mathbf{x}\right)= & \frac{df}{dx}\left(\mathbf{x}\right)\hat{\mathbf{u}}_{k}+\mathbf{w}_{k}
\end{align*}
To simplify this we denote:
\begin{align*}
	\hat{\mathbf{U}}= & \left[\hat{\mathbf{u}}_{1}|\cdots|\hat{\mathbf{u}}_{k}\right]\\
	\mathbf{D}= & \left[f\left(\mathbf{x}+\hat{\mathbf{u}}_{1}\right)-f\left(\mathbf{x}\right)|\cdots|f\left(\mathbf{x}+\hat{\mathbf{u}}_{k}\right)-y\left(\mathbf{x}\right)\right]
\end{align*}
And get:
\[
\mathbf{D}=\frac{df}{dx}\left(\mathbf{x}\right)\hat{\mathbf{U}}
\]

If the vectors $\hat{\mathbf{u}}_{1},\hat{\mathbf{u}}_{2},\dots,\hat{\mathbf{u}}_{k}$ span the $n$-dimensional intrinsic space (i.e. the matrix $\hat{\mathbf{U}}$ has a full row rank), we can invert this relationship using the pseudo-inverse of $\hat{\mathbf{U}}$ matrix in order to recover the observation function Jacobian by:
\[
\mathbf{D}\hat{\mathbf{U}}^{T}\left[\hat{\mathbf{U}}\hat{\mathbf{U}}^{T}\right]^{-1}=\frac{df}{dx}\left(\mathbf{x}\right)
\]
This suggests a method for estimating the Jacobian of an unknown observation function, using an array of measurements with a known structure. $\mathbf{M}\left(\mathbf{y}\right)$ can then be estimated by:
\begin{equation}
\mathbf{M}\left(\mathbf{y}\right)=\frac{df}{dx}\left(\mathbf{x}\right)\frac{df}{dx}\left(\mathbf{x}\right)^{T}\approx\mathbf{D}\hat{\mathbf{U}}^{T}\left[\hat{\mathbf{U}}\hat{\mathbf{U}}^{T}\right]^{-1}\left[\hat{\mathbf{U}}\hat{\mathbf{U}}^{T}\right]^{-T}\hat{\mathbf{U}}\mathbf{D}\label{eq:array-est}
\end{equation}
We notice that if this array is rotated around the point $\mathbf{x}$ with a rotation matrix $\mathbf{R}$ we get a rotation of the estimated Jacobian:
\begin{align*}
	\mathbf{D}\left[\mathbf{R}\hat{\mathbf{U}}\right]^{T}\left[\mathbf{R}\hat{\mathbf{U}}\left[\mathbf{R}\hat{\mathbf{U}}\right]^{T}\right]^{-1} & =\mathbf{D}\hat{\mathbf{U}}^{T}\mathbf{R}^{T}\left[\mathbf{R}\hat{\mathbf{U}}\hat{\mathbf{U}}^{T}\mathbf{R}^{T}\right]^{-1}\\
	& =\mathbf{D}\hat{\mathbf{U}}^{T}\mathbf{R}^{T}\mathbf{R}^{-T}\left[\hat{\mathbf{U}}\hat{\mathbf{U}}^{T}\right]^{-1}\mathbf{R}^{-1}\\
	& =\mathbf{D}\hat{\mathbf{U}}^{T}\left[\hat{\mathbf{U}}\hat{\mathbf{U}}^{T}\right]^{-1}\mathbf{R}^{-1}\\
	& =\frac{df}{dx}\left(\mathbf{x}\right)\mathbf{R}^{-1}
\end{align*}
However the estimation of $\mathbf{M}\left(\mathbf{y}\right)$ in \cref{eq:array-est} is unaffected by this rotation:
\begin{equation}
\frac{df}{dx}\left(\mathbf{x}\right)\mathbf{R}^{-1}\left[\frac{df}{dx}\left(\mathbf{x}\right)\mathbf{R}^{-1}\right]^{T}=\frac{df}{dx}\left(\mathbf{x}\right)\mathbf{R}^{-1}\mathbf{R}^{-T}\cdot\left[\frac{df}{dx}\left(\mathbf{x}\right)\right]=\frac{df}{dx}\left(\mathbf{x}\right)\frac{df}{dx}\left(\mathbf{x}\right)^{T}
\end{equation}
This shows that the estimation suggested in (\cref{eq:array-est}) is invariant to a rigid rotation of the measurement directions $\hat{\mathbf{u}}_{1},\ldots,\hat{\mathbf{u}}_{n}$.
This invariance to a rotation makes this setting practical for intrinsic metric estimation, since one does not have to align the sensor array in any particular way with respect to the intrinsic axis and one can even allow this array to rotate between observations at different points. \cref{fig:Observation-with-a} shows the set of points observed using a sensor array of 3 directions (4 total measurements) for 10 points, where the intrinsic manifold $\mathcal{X}$ is a disk in 2-dimensional Euclidean space.

\begin{figure}[h]
	\begin{subfigure}[b]{0.48\columnwidth}%
			\includegraphics[width=1\textwidth]{figures/Chapter_4/sensor_array_example}
	\end{subfigure}\hfill
	\begin{subfigure}[b]{0.48\columnwidth}%
			\includegraphics[width=1\textwidth]{figures/Chapter_4/sensor_array_example_intrinic}
	\end{subfigure}
	\caption{Observation with a rigid sensor array Observation for $10$ representative points \label{fig:Observation-with-a}}
\end{figure}

While observation in $n$ linearly independent directions is sufficient in theory in order to estimate the function Jacobian, observation in more directions will make the estimation more robust in the presence of possible observation noise.

\subsection{Intrinsic isotropic Gaussian mixture model \label{ssec:Intrinsic-isotropic-GMM}}

Assume that the intrinsic data are sampled from an $n$-dimensional \ac{GMM} consisting of $N$ isotropic Gaussian distributions, all with known variance $\sigma_{int}^{2}$, where the means of these distributions are points on the intrinsic manifold which are denoted by $\left\{ \mathbf{x}_{i}\right\} _{i=1}^{N}=\mathcal{X}_{S}=\in\mathcal{X}$. $N_{i}$ points are sampled from the $i$-th Gaussian and we denote these points using a double index:
\[
{\bf x}_{i,j}\sim\mathcal{N}\left(\mathbf{x}_{i},\sigma_{int}^{2}I\right)
\]
This model is illustrated in \cref{fig:Intrinsic-Gmm-setting}. Sampled points are observed via an observation function $f:\mathbb{R}^{n}\to\mathbb{R}^{m}$ and are distorted by an \ac{AWGN} with variance $\sigma_{obs}^{2}$ introduced in the observation process resulting in observed data points:
\[
{\bf y}_{i,j}=f\left({\bf x}_{i,j}\right)+w_{i,j}
\]
where $w_{i,j}\sim\mathcal{N}\left(0,\sigma_{obs}^{2}\mathbf{I}\right)$. If the variance $\sigma_{int}^{2}$ of the intrinsic Gaussians is small with respect to the higher-derivatives of the observation function, each Gaussian cluster will only effectively experiences the local linear part of the observation function and one can regard the observed cluster generated by an intrinsic Gaussian as a Gaussian distribution with a transformed covariance and mean:.
\begin{equation}
{\bf y}_{i,j}=f\left({\bf x}_{i,j}\right)+w_{i,j}\sim\mathcal{N}\left(f\left(\mathbf{x}_{i}\right),\sigma_{int}^{2}\frac{df}{dx}\left(\mathbf{x}_{i}\right)\frac{df}{dx}\left(\mathbf{x}_{i}\right)^{T}+\sigma_{obs}^{2}\mathbf{I}\right)\label{eq:prob-pca}
\end{equation}
We see that the observed data also has the structure of a \ac{GMM} , however, the covariance of the observed Gaussians are anisotropic due to stretching and contraction induced by the observation function Jacobian. The distribution of a single observed Gaussian corresponds to the model presented in the probabilistic \ac{PCA} interpretation \cite{tipping1999probabilistic} where it is shown that the optimal estimator, in the maximum-likelihood sense, of the matrix $\frac{df}{dx}\left(\mathbf{x}_{i}\right)\frac{df}{dx}\left(\mathbf{x}_{i}\right)^{T}$ can be computed by performing \ac{PCA} and finding the $n$ most significant principle directions of the observed sample covariance
matrix $\mathbf{S}_{i}$:
\[
\begin{aligned}{\bf \bar{y}}_{i} & =\frac{1}{N_{i}}\sum_{j=1}^{N_{i}}{\bf y}_{i,j}\\
\mathbf{S}_{i} & =\frac{1}{N_{i}}\sum_{j=1}^{N_{i}}\left[{\bf y}_{i,j}-{\bf \bar{y}}_{i}\right]\left[{\bf y}_{i,j}-{\bf \bar{y}}_{i}\right]^{T}
\end{aligned}
\]
This matrix can be decomposed using spectral decomposition:
\[
\mathbf{S}_{i}=\left[\mathbf{u}_{1}\mathbf{u}_{2}\cdots\mathbf{u}_{n}\cdots\mathbf{u}_{m}\right]\left[\begin{array}{ccccc}
\sigma_{1}^{2} & 0 & 0 & \ldots & 0\\
0 & \ddots & 0 & \ldots & \vdots\\
0 & 0 & \sigma_{n}^{2} & \ldots & 0\\
\vdots & \vdots & \vdots & \ddots & \vdots\\
0 & \ldots & 0 & \ldots & \sigma_{m}^{2}
\end{array}\right]\left[\begin{array}{c}
\mathbf{u}_{1}^{T}\\
\mathbf{u}_{2}^{T}\\
\vdots\\
\vdots\\
\mathbf{u}_{n}^{T}\\
\vdots\\
\mathbf{u}_{m}^{T}
\end{array}\right]
\]
And the maximum-likelihood estimator is given by \cite{tipping1999probabilistic}:
\begin{equation}
\begin{aligned}\frac{df}{dx}\left(\mathbf{x}_{i}\right)\frac{df}{dx}\left(\mathbf{x}_{i}\right)^{T} & \approx\left[\mathbf{u}_{1}\mathbf{u}_{2}\cdots\mathbf{u}_{n}\right]\left[\begin{array}{ccccc}
\sigma_{1}^{2}-\bar{\sigma^{2}} & 0 & 0 & \ldots & 0\\
0 & \ddots & 0 & \ldots & \vdots\\
0 & 0 & \sigma_{n}^{2}-\bar{\sigma^{2}} & \ldots & 0\\
\vdots & \vdots & \vdots & 0 & \vdots\\
0 & \ldots & 0 & \ldots & 0
\end{array}\right]\left[\begin{array}{c}
\mathbf{u}_{1}^{T}\\
\mathbf{u}_{2}^{T}\\
\vdots\\
\vdots\\
\mathbf{u}_{n}^{T}
\end{array}\right]\\
\bar{\sigma^{2}} & =\frac{1}{m-n}\sum_{k=n+1}^{m}\sigma_{k}^{2}
\end{aligned}
\end{equation}
Assuming that we know the clustering structure of observed measurements (i.e. we know which observation points originate from the same Gaussian)
we can estimate $\mathbf{M}\left({\bf y}_{i}\right)=\frac{df}{dx}\left(\mathbf{x}_{i}\right)\frac{df}{dx}\left(\mathbf{x}_{i}\right)^{T}$ for each observed Gaussian separately. This setting is similar to the setting presented in \cite{singer2008non} where short bursts of diffusion processes with isotropic diffusion were used, resulting in an approximate intrinsic \ac{GMM} process and leading to the same estimator of the intrinsic metric using local application of \ac{PCA}.

\begin{figure}[h]
	\begin{centering}
		\includegraphics[width=0.6\textwidth]{figures/Chapter_4/intrinsic_GMM_model}
	\end{centering}
	\caption[Intrinsic \ac{GMM} setting]{Intrinsic \ac{GMM} setting. $10$ representative Gaussians from the mixture model are visualized in the intrinsic space for the case where $\mathcal{X}$ is a disk in 2-dimensional Euclidean space \label{fig:Intrinsic-Gmm-setting}}
\end{figure}

\subsection{Intrinsic diffusion process\label{ssec:Intrinsic-diffusion-processes}}

Data often originates from observation of a latent dynamical system over time. In such situations, information is not encoded only in the measurement values but also in time at which these measurements were received. Prior knowledge about the statistical model of the latent system combined with this temporal information can be used in order to estimate the intrinsic metric. 

Here we present a setting where a latent system governed by a single continuous diffusion process, as is encountered in many scientific fields, allows one to estimate the local intrinsic metric from the observed measurement process. Consider an $n$-dimensional Itô process governed by the following \ac{SDE} :
\[
d\mathbf{x}\left(t\right)=\mu_{int}\left(\mathbf{x}\left(t\right)\right)\cdot dt+\sigma_{int}dB\left(t\right)
\]
A scalar observation function $f:\mathbb{R}^{n}\to\mathbb{R}$ is
applied to this process creating the observation process $f\left(\mathbf{x}\left(t\right)\right)$
. According to Itô's lemma \cite{karatzas2012brownian} we get that
the observed process propagates according to the following \ac{SDE}:
\[
df\left(\mathbf{x}\left(t\right)\right)=\left(\frac{df}{dx}\left(\mathbf{x}\left(t\right)\right)\mu_{int}\left(\mathbf{x}\left(t\right)\right)+\text{Tr}\left(\frac{d^{2}f}{dx^{2}}\left(\mathbf{x}\left(t\right)\right)\frac{\sigma_{int}^{2}}{2}\right)\right)dt+\sigma_{int}\frac{df}{dx}\left(\mathbf{x}\left(t\right)\right)dB\left(t\right)
\]
To simplify this equation we denote:
\[
\mathbf{\mu}_{obs}\left(\mathbf{x}\left(t\right)\right)=\frac{df}{dx}\left(\mathbf{x}\left(t\right)\right)\mu_{int}\left(\mathbf{x}\left(t\right)\right)+\text{Tr}\left(\frac{d^{2}f}{dx^{2}}\left(\mathbf{x}\left(t\right)\right)\frac{\sigma_{int}^{2}}{2}\right)
\]
And get:
\[
df\left(\mathbf{x}\left(t\right)\right)=\mathbf{\mu}_{obs}\left(\mathbf{x}\left(t\right)\right)dt+\sigma_{int}\frac{df}{dx}\left(\mathbf{x}\left(t\right)\right)dB\left(t\right)
\]
We can easily consider the case where the observation function is a $m$-dimensional vector function $f:\mathbb{R}^{n}\to\mathbb{R}^{m}$ which leads to a set of such equations for each dimension that can be written in matrix form:
\[
df\left(\mathbf{x}\left(t\right)\right)=\mathbf{\mu}_{obs}\left(\mathbf{x}\left(t\right)\right)dt+\sigma_{int}\frac{df}{dx}\left(\mathbf{x}\left(t\right)\right)dB\left(t\right)
\]
An observation of such an intrinsic diffusion process is described in \cref{fig:Diffusion-process-observation}.
\begin{figure}[h]
	\centering{}\includegraphics[width=1\textwidth]{figures/Chapter_4/diff_set}\caption{Diffusion process observation\label{fig:Diffusion-process-observation}}
\end{figure}
Observing this process at equally spaced time-intervals results in a discreet process which can be, according to the Euler–Maruyama schema\cite{kloeden1992numerical}, approximately described by the following discrete stochastic difference equation: 
\[
f\left(\mathbf{x}\left[k+1\right]\right)=f\left(\mathbf{x}\left[k\right]\right)+\mathbf{\mu}_{obs}\left(\mathbf{x}\left[k\right]\right)\triangle t+\sigma_{int}\frac{df}{dx}\left(\mathbf{x}\left[k\right]\right)W\left[k\right]\triangle t
\]
Where $\triangle t$ is the sampling interval and $W\left[k\right]$
is a discrete white Gaussian noise process with unit variance. To
further simplify this we re-denote:
\begin{align*}
	\mu_{obs}\left(\mathbf{x}\left[k\right]\right)= & \mathbf{\mu}_{obs}\left(\mathbf{x}\left[k\right]\right)\triangle t\\
	\sigma_{int}= & \sigma_{int}\triangle t
\end{align*}
And get the following difference equation:
\[
f\left(\mathbf{x}\left[k+1\right]\right)-f\left(\mathbf{x}\left[k\right]\right)=\mathbf{\mu}_{obs}\left(\mathbf{x}\left[k\right]\right)+\sigma_{int}\frac{df}{dx}\left(\mathbf{x}\left[k\right]\right)W\left[k\right]
\]
This states that the difference process of the discretely sampled observation process has an approximately normal distribution with the following distribution:
\begin{equation}
f\left(\mathbf{x}\left[k+1\right]\right)-f\left(\mathbf{x}\left[k\right]\right)\sim\mathcal{N}\left(\mathbf{\mu}_{obs}\left(\mathbf{x}\left[k\right]\right),\sigma_{int}^{2}\frac{df}{dx}\left(\mathbf{x}\left[k\right]\right)\frac{df}{dx}\left(\mathbf{x}\left[k\right]\right)^{T}\right)\label{eq:dist_diff}
\end{equation}
We can also consider the possibility \ac{AWGN} being introduced in the observation process and get the following equation for the observation process:
\begin{equation}
\mathbf{y}\left[k+1\right]-\mathbf{y}\left[k\right]\sim\mathcal{N}\left(\mathbf{\mu}_{obs}\left(\mathbf{x}\left[k\right]\right),\sigma_{int}^{2}\frac{df}{dx}\left(\mathbf{x}\left[k\right]\right)\frac{df}{dx}\left(\mathbf{x}\left[k\right]\right)^{T}+\sigma_{obs}^{2}\mathbf{I}\right)
\end{equation}
This indicates that the difference process can be seen as being sampled from a Gaussian distribution whose covariance and drift are location dependent. A straightforward method to estimate $\mathbf{M}\left(\mathbf{y}_{i}\right)$ for a specific observation point $\mathbf{y}_{i}$ is to look for all observation time indexes $k$ for which $\mathbf{y}[k]\approx\mathbf{y}_{i}$ and donate this set by $K_{\mathbf{y_{i}}}$:
\[
K_{\mathbf{y_{i}}}=\left\{ k|\mathbf{y}[k]\approx\mathbf{y}_{i}\right\} 
\]

If the observation function is sufficiently smooth and invertible, observations which are close enough to each other in the observation space originate from similar intrinsic points and therefor experience similar observation function Jacobians. As a result the observed process ``jumps'' $f\left(\mathbf{x}[k+1]\right)-f\left(\mathbf{x}[k]\right)$ for all $k\in K_{\mathbf{y_{i}}}$ will be distributed with very similar Gaussian distributions. \cref{fig:Process-jump-clustering} illustrates how three points from the observed process which are ``close-enough'' to each other in the observation space can be clustered together. The three process ``jumps'' originating from these three points are all approximately distributed with the same distribution. Gathering enough such ``jumps'' from a single location allows us to estimate the covariance matrix and the intrinsic metric again by performing local \ac{PCA} as described in \cref{ssec:Intrinsic-isotropic-GMM} and \cite{tipping1999probabilistic}. 

\begin{figure}[h]
	\centering{}\includegraphics[width=1\textwidth]{figures/Chapter_4/cluster_for_metric}\caption{Process jump clustering \label{fig:Process-jump-clustering}}
\end{figure}

This method clusters together vectors of the observed difference process which are almost identically distributed according to (\cref{eq:dist_diff}). If these clusters are disjoint this results in independent Gaussians distributions and leads again to an observed \ac{GMM} as in \cref{ssec:Intrinsic-isotropic-GMM}. However, in general this clusters are not necessarily disjoint and the different Gaussian distributions are not independent of each other, in such cases this estimator can only be regarded as a heuristic approximation of the true maximum-likelihood estimator.

Diffusion processes were used in \cite{singer2008non} in order to expose the intrinsic metric, however they were used in a different setting where multiple burst of intrinsic diffusion processes starting from the same initial state were used, a setting which is less realistic and is applicable mainly for dimensionality reduction of complex non-linear systems which admit simulations as in 




	
	\bibliographystyle{siamplain}
	\bibliography{references}
	
\end{document}