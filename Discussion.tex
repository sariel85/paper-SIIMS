\externaldocument{paper.tex}
	\section{Conclusions}
		\label{sec:conclusions}
		
		In this work, we addressed the subject of intrinsic and isometric
	manifold learning. We first showed that the need for methods, which
	preserve the latent Euclidean geometric structure of data manifolds,
	arises naturally when inherently low-dimensional systems are observed
	via high-dimensional non-linear measurements or observations. We presented
	a new manifold learning algorithm which uses local properties of the
	observation function to estimate a local intrinsic metric (the ``push-forward''
	metric of the observation function). This metric was then used to
	estimate intrinsic geometric proprieties of the data directly from
	the observed manifold as if they were calculated in the low-dimensional
	latent space. We discussed a few settings under which estimation of
	the required local observation function properties is possible from
	the observed data itself. Unfortunately, we recognized that due to
	their local nature, these metric estimation methods are not sufficiently
	robust to high curvature of the observation function as well as to
	noise. We suggested to overcome this by parameterizing all estimated
	intrinsic metrics on the observed manifold as the output of a single
	\ac{ANN} and performing a signal optimization in order to estimate
	all local intrinsic metric simultaneously. This procedure has probabilistic
	reasoning and was shown to be equivalent to maximum-likelihood estimation
	under a certain statistical model. We showed that this couples the
	metric estimations at different points on the manifold, regularizes
	the estimation and imposes smoothness of the estimated metric. We
	discussed the possibility of additionally imposing regularization
	on the estimation by the net structure and weight decay terms, which
	is common practice in \acp{ANN}. By combining a robust intrinsic
	metric estimation method and an algorithm which can use these metrics
	to build an intrinsic and isometric embedding, we devised an algorithm,
	which can automatically recover the geometric structure of a latent
	low-dimensional manifold from its observations via an unknown non-linear
	high-dimensional function. Finally we focused on the example of mapping
	and positioning an agent using a sensor network of unknown nature
	and modalities. We showed that our proposed algorithm can recover
	the structure of the space in which the agent moves and can correctly
	position the agent within that space. Due to the intrinsic nature
	of our method, it can perform this mapping and positioning without
	the need for prior knowledge of a measurement model to explain the
	connection between the observed measurements and the position of the
	agent. This invariance to the type of measurement used, was shown
	to be suitable in a setting such as indoor positioning where the exact
	measurement model is usually unknown.
	
	It is evident that our method outperforms the local metric estimation
	approach for all the tested examples and enables us to learn intrinsically-isometric
	representation for broader cases problems given very sparse sampling
	and observation noise of the same scale of the intrinsic variance
	which is used to estimate the intrinsic metric. 
	
	The use of neural networks as regression functions. provides a powerful
	regularizing factor in intrinsic metric estimation. Their general
	structure with the plethora of optimization algorithms, regularization
	tweaks and efficient implementation methods make them a very good
	candidate for a parametric function family to be used in the problem
	of metric estimation.
	
	The estimation method described in this chapter was proven to have
	probabilistic reasoning when assuming an intrinsic-isometric \ac{GMM},
	however it also has an intuitive interpretation in a more general
	case. As can be observed in \cref{eq:KL}, the network tries approximate
	locally calculated estimations of the intrinsic metric while adhering
	to a globally smooth and regularized model. This can serve as a good
	general heuristic method to impose global smoothness and regularization
	on locally estimated intrinsic metrics under other probabilistic or
	non-probabilistic setting such as the one presented in
	\cref{ssec:Intrinsic-isotropic-GMM}.
	
	A somewhat similar method was presented in \cite{bengio2004non} where
	an \ac{ANN} was also used in order to parametrize the spanning vectors
	of the local tangent plane to a manifold. The only criterion provided
	for the estimation was minimization of the distance of observed points
	from the local tangent plane. The estimation was not given probabilistic
	reasoning and the estimated spanning vector had no physical meaning
	and did not provide any metric interpretation of information. In our
	work, we not only retrieve the tangent plane to the manifold at each
	point but also an intrinsic metric and we do so with probabilistic
	reasoning.
	
	In the case of neural networks ``the devil is in the details'' and
	implementation plays a large role in the final performance of the
	trained network and the subjects of neural network training and regularization
	are vast topics which are out of the scope of this work. Our main
	contribution is to show that neural networks can, in principle be
	used, to regularize estimations and in this specific case intrinsic
	metric estimation and we do not attempt to claim that this architecture
	and optimization procedure achieve an optimal solution.
	
	The global estimation approach presented in this chapter has some
	secondary advantages in addition to providing additional robustness
	to the intrinsic metric estimation when compared to local metric estimation
	methods. As opposed to local methods which only estimate the intrinsic
	metric for observed sample points, the regression approach produces
	an estimation of the intrinsic metric on the whole observed space.
	This can be possibly used to generate additional points in between
	existing points, thus artificially increasing the sample density.
	This can improve both the short-range intrinsic distance estimation
	described in \cref{ssec:Intrinsic-geometry-approximation} and
	the geodesic distance estimation described in \cref{ssec:Global-geometry-approximation}.
	Both effects should improve the results of the algorithm presented
	in \cref{sec:Intrinsic-isometric-manifold-learning}.
	
	In \cite{bengio2004non} it was shown that by leaning the tangent
	space to the manifold at each point, one can ``walk'' on the manifold
	by making infinitesimal steps each time in the tangent space. Since
	with the method suggested in this chapter, we do not only estimate
	the tangent space but the intrinsic metric as well, we know how far
	we have gone in terms of the distance, an ability that might be relevant
	to applications such as non-linear interpolation \cite{bregler1995nonlinear}.
	
	
	\subsection{Pre-processing}
		\label{subsec:Pre-processing}
	
	Since our approach does not assume anything about the structure of
	the observed data, one can perform pre-processing stages for reducing
	the dimensionality of the data and possibly removing noise without
	worrying about maintaining the structure of the data. This has the
	advantage of enabling the use of other existing dimensionality reduction
	methods as pre-processing stages to our algorithm. For example, when
	localization and mapping were performed using image data, the algorithm
	did not treat the data as images, which allowed us to use \ac{PCA}
	to lower the dimensionality of the image data greatly. This would
	not be possible for vision based algorithms which exploit the image
	structure since the application of \ac{PCA} would remove this structure
	and such algorithm could no longer be applied. For our method the
	image structure is unimportant and one can do without it.
	
	\subsection{Practical application to indoor-positioning}
		\label{subsec:Practical-application-to}
	
	The ability to localize sets of measurements without knowledge of
	the model connection positions and measurements is a problem encountered
	in real life indoor positioning due to the complexity of the models
	and the variations, as described in \cref{sec:introduction}.
	Learning a model for each specific setting requires large amount of
	labeled data. Our algorithm might replace the need for labeled data
	acquisition with a simple prior assumption on the unlabeled measurement
	acquisition process. 
	
	This has been showed to work in theory however it has yet to be applied
	to a realistic indoor positioning system. Basic physical experiments
	using a randomly walking agent in 2-dimensional space are underway.
	In these experiments a robotic iRobot Roomba programmable robotic
	vacuum cleaner was controlled by a Raspberry-Pi mini computer and
	performed a random walk inside a 2-dimensional region. Signal acquisition
	was performed either my measuring the \ac{RSS} from WiFi stations
	or via a wide angle lens which produced 360 degree images. Results
	from the experiments will be published when concluded. Images from
	these experiments are shown in \cref{fig:Roomba-experiments}.
	
	\iffalse
	\begin{figure}[h]
		\begin{centering}
			\begin{minipage}[t]{0.45\columnwidth}%
				\begin{flushleft}
					\subfloat[\ac{RSS} measurement]{\begin{centering}
							\includegraphics[width=1\textwidth]{figures/Chapter_6/Roomba_wifi}
							\par\end{centering}
					}
					\par\end{flushleft}%
			\end{minipage}\hfill{}%
			\begin{minipage}[t]{0.45\columnwidth}%
				\subfloat[Panoramic camera]{\begin{centering}
						\includegraphics[width=1\textwidth]{figures/Chapter_6/Roomba_pano}
						\par\end{centering}
				}%
			\end{minipage}
			\par\end{centering}
		\caption{Roomba experiments \label{fig:Roomba-experiments}}
	\end{figure}
	\fi
	
	
	\subsection{Generalized localization problems}
	
	Throughout this work and in \cref{sec:introduction} and
	\cref{ssec:localization} specifically, we discussed the problem
	of localization in sensor networks and presented it as an example
	of a problem which requires an intrinsic and isometric manifold learning
	method in order to fully utilize the observed data and recover a mapping
	of the physical space and a localization of an agent within this space.
	However, this should not be regarded as a specific problem (although
	it is by itself an important one) but as a general prototypical problem,
	for which physical localization in 2-dimensional or 3-dimensional,
	is just one intuitive manifestation. Indeed, any problem where the
	uncovering of the global geometric structure of a low-dimensional
	parameter space is desirable, can be regarded as a ``localization
	`` problem. Trivially, this can occur when structure recovery is
	by itself a goal and one can think of many examples where the global
	structure of the parameter space is important or has some special
	meaning (for example for imaging or molecule structure recovery),
	yet this global structure is also crucial when one requires the intrinsic
	structure of a low-dimensional vector space. The process of observation
	via a non-linear function causes the data to lose its global structure,
	and the initially ``flat'' manifold becomes curved. This hinders
	our ability to perform operations on the parameter space which require
	a vector space structure such as addition, subtraction, division and
	multiplication. This are the basic operation required for more high-level
	processing such as averaging, clustering, interpolation, learning
	using $k$-NN (especially when samples are sparse and neighbors are
	far from each other) etc. Embedding the data back into a low-dimensional
	space recovers the vector space structure of the original latent space
	and enables these operations.
	
	\subsection{Automatic labeled data acquisition}
	
	Supervised machine learning uses labeled data (consisting of pairs
	of input objects and desired output values) and attempts to ``learn''
	or infer a functional connection between the two. Learning using more
	labeled data increases the learners ability to unseen examples and
	usually leads to better performance of the trained model. Unfortunately
	acquisition of labeled data is usually non-trivial, expensive and
	requires an already existing method to correctly label data. Our proposed
	approach can be seen as a method for automatic acquisition of labels
	as it operates in a completely unsupervised manner and produces data
	labeling as an output. The ability to produce labeled data can thus
	be reduced to the task of exploring the parameter space of the system
	under some restriction or statistical model which allows us to calculate
	a local intrinsic metric as described in this work. 
	
	This automatic acquisition of labeled data can be much easier in many
	cases than producing labeled data using an existing labeling method.
	As an example we return to the localization example discussed in depth
	in \cref{ssec:localization}. To acquire labeled data in
	the described setting, one would be required to perform a large set
	of observations (color images, depth maps, \ac{RSS} values etc.)
	and provide for each such observation the 2-dimensional coordinates
	at which they were taken. This procedure requires time, patience and
	an existing way to measure the correct coordinates. This becomes even
	more complicated if one considers localization in 3-dimensional space
	where accurate positioning requires special equipment. Alternatively,
	with our approach, one could use a randomly waking agent or a simple
	sensor array (as described in \cref{sec:Intrinsic-Metric-Estimation})
	to acquire a large set of unlabeled data, and then use our proposed
	algorithm to retrieve an approximation of their labels. One can then
	use a supervised machine learning algorithm in order to infer the
	label of new observations. In the application described in \cref{ssec:localization}, once labeling of a large set of images
	from the apartment space is acquired using our algorithm, one can
	feed these to a vision based algorithm which can be robust to variation
	in appearance of the space, such as changes of illumination, obstructions,
	or even some changes in the structure of the apartment due to movement
	of some objects.
	
	\subsection{Holistic approach}
	
	The algorithm presented in this work suggests a two-stage solution
	for intrinsic-isometric embedding of manifolds in low-dimensional
	Euclidean space:
	\begin{enumerate}
		\item Recovering local intrinsic metrics using prior-knowledge and assumptions
		about the intrinsic process and using them to calculate intrinsic
		inter-point distances.
		\item Constructing and embedding of the observed points into a low-dimensional
		space, where the Euclidean distance respects the calculated intrinsic
		inter-point distances.
	\end{enumerate}
	If the first stage of this approach does not result in accurate enough
	estimations of the intrinsic geometry, the second stage can potentially
	fail. Combining the two stages described above into a one stage embedding
	method could be beneficial. A possible way to implement this is to
	directly learn an embedding function of the observed data into a low-dimensional
	space such that the embedding best fits prior-knowledge or assumption
	we have about the intrinsic system. This could be implemented for
	example with an \ac{ANN} for which the cost function will be the
	likelihood that the embedding originated from the statistical model
	of the intrinsic latent system. Such an implementation would however,
	rarely converge to minima which represent a good embedding of the
	data without being given an initial solution which is close enough
	to the true structure. One can however use the embedding received
	via the presented two stage approach as an initial target embedding
	for such an \ac{ANN}, and then further maximize the likelihood of
	the embedding directly on the embedding space. 
	
	\subsection{Sensor fusion}
	
	The fact that our proposed algorithm is invariant to the observation
	function, allows sensor fusion. Measurements from different sensors
	modalities can be simply concatenated, creating a higher-dimensional
	observation function. Although this should work in theory, it is clear
	that if the sensors are improperly scaled with respect to each other
	this will not be optimal. Further research should be invested in considering
	the proper way to fuse different measurement modalities in a way that
	optimizes the results of the algorithm. 
	
	\subsection{Towards data-driven Taken's embedding}
	
	Our algorithm attempts to produce an embedding of the observed manifold
	into a low-dimensional space, where similar observations are embedded
	to similar locations in the constructed embedding space. This causes
	a problem if different intrinsic points have identical or very similar
	observation values. In this work we resolved this by assuming that
	the observation function is invertible. One practical approach to
	deal with this situation is to add observations until any such similar
	points are distinguished. However, when a set of observations is given
	and one cannot add additional ``physical'' observations we require
	a different way required. If temporal information about the dynamical
	observation process is available one can, in principle, distinguish
	points by their respective past and future observations. Such a separation
	between points would be especially beneficial in the presence of high
	levels of observation noise which might make originally distinct measurement
	values similar to each other. Our proposed algorithm currently uses
	temporal side-information only for the stage of estimating the intrinsic
	metric and does not incorporate this information for the purpose of
	the final embedding into low-dimensional space. One possible approach
	for incorporating this information naturally into our algorithm is
	by the observation of several lagged measurements simultaneously,
	by looking far ``enough'' into the past, such that any two points
	become distinguishable. This was proven for the case of deterministic
	latent dynamical system by Taken's embedding theorem \cite{takens1981detecting}.
	
	\subsection{Implementation using vector-diffusion maps}
	
	Our proposed manifold learning algorithm computes global intrinsic
	geometric properties by using the shortest-path algorithm in order
	to approximate inter-point geodesic distances. The shortest-path algorithm
	operated by propagating the distance function from a source point
	across the manifold. A similar process of propagating local vector
	information (instead of scalar distance information) could be used
	to allow for direct long range intrinsic Euclidean distance estimation.
	A framework for vector diffusion on Riemannian manifold has been presented
	in \cite{singer2012vector} and could possibly be used in order to
	implement this. Euclidean distance estimation by vector diffusion
	should be unaffected by non-convexity of the intrinsic manifold and
	can allow for use of global distances making the final embedding more
	robust.
	
	\subsection{Incorporating labeled data}
		\label{subsec:Incorporating-labeled-data}
	
	The algorithm proposed in this work operates in a completely unsupervised
	manner. The only prior-knowledge used is the assumption about the
	intrinsic data structure, which allows for estimation of the local
	intrinsic metric. While this is a desired property of the algorithm,
	since labeled data is usually harder to gather relative to unlabeled
	data, it does not harness the advantages of labeled data if such data
	does exist. We mention two possible ways to incorporate existing labeled
	data. Labeled data can be used as part of the intrinsic metric learning,
	where similar settings have been used in existing works in order to
	generate metrics which best describe existing label variations on
	data manifold \cite{yang2006distance,xing2003distance}. Alternatively,
	labeled data can be used in the embedding procedure itself, where
	one can easily add additional distance restriction known from labeled
	data. In the localization example given in \cref{ssec:localization}
	for example, we notice a slight distortion in the embedding on a global,
	long-range scale. This is because we only incorporate local distance
	information in the final embedding optimization, which does not fully
	restrict the global conformation in practical noisy situations. The
	short-distance estimations have errors, which induce slight distortions
	in the embedding. These distortions are not significant on small scales
	but they accumulate on larger scales and longer distances. Adding
	a few robustly measured long distances can be extremely beneficial
	for correcting this and even a very low number of such constraints
	could stabilize the embedding considerably, removing this accumulated
	error. Incorporating such robustly measured distances in the algorithm
	suggested in \cref{sec:Intrinsic-isometric-manifold-learning}
	can be done easily by giving such distance constraints much larger
	weights in the stress function presented in (\cref{eq:w-intrinisc_stress})
	than those of distances estimated from the observed data, which are
	less reliable.